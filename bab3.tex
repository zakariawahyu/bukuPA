%---------------------------------------------------------------
\chapter{\babTiga}
%---------------------------------------------------------------

%-----------------------------------------------------------------------------%
\section{Sistem Arsitektur}
%-----------------------------------------------------------------------------%

Perancangan sistem arsitektur aplikasi "\textbf{SiPJabS : Sistem Pengawakan Jabatan Struktural}" dapat dilihat pada \textbf{Gambar 3.1.} berikut: 

\begin{figure}
	\centering
	\includegraphics[width=1.1\textwidth]
	{pics/LowLevelDiagram.png}
	\caption{\textit{Low Level Design}}
	\label{fig:31}
\end{figure}


\subsection{Gambaran Umum Sistem}

Aplikasi "\textbf{SiPJabS : Sistem Pengawakan Jabatan Struktural}" merupakan aplikasi berbasis web yang memudahkan bagi perusahaan dalam pencarian seorang kandidat atau posisi yang kosong. Dalam pembuatan aplikasi ini dibutuhkan fitur \textit{filtering} yang digunakan untuk pencarian kandidat baru, yang sesuai dengan ketentuan yang sudah ditetapkan oleh perusahaan. Sistem \textit{filtering} dapat dilakukan setiap saat, untuk menggantikan pekerjaan lama yang telah berhenti dikarenakan pensiun, meninggal, mengundurkan diri atau diberhentikan karena suatu kebijakan tertentu. \\

Data-data pegawai yang berada di Universitas Telkom dapat dilihat dan data tersebut bersifat rahasia. Sehingga aplikasi ”\textbf{SiPJabS : Sistem Pengawakan Jabatan Struktural}” hanya dapat diakses oleh orang tertentu. Aplikasi ini terdapat satu \textit{user} yang dapat mengelola proses \textit{filtering} dan satu \textit{admin} yang mengelola infrastruktur \textit{database} dan proyek \textit{server} serta jaringan.  

Sistem \textit{filtering} pada apliaksi ini terbagi menjadi dua bagian, yang pertama merupakan \textit{fitering} secara umum dengan isi \textit{form} seperti jabatan minimal dan masa kerja. Yang kedua merupakan \textit{filtering} secara khusus, dimana \textit{user} dapat mencari kandidat dengan syarat yang lebih spesifik lagi untuk dijadikan pilihan, kemudian akan terdapat beberapa nama kandidat, apabila sudah menentukan pilihan dapat menekan tombol \textit{button} pada nama yang akan dipilih dan akan masuk dalam kandidat sementara.

Apabila proses pencarian kandidat sudah ditemukan dengan salah satu proses \textit{filtering} yang sudah dijelaskan diatas maka, proses selanjutnya akan masuk dalam pembuatan berita acara dan dapat dicetak berupa file pdf.  

\subsection{Target Pengguna Aplikasi}

Aplikasi \textbf{SiPJabS} memiliki beberapa target pengguna diantaranya sebagai berikut:

\begin{enumerate}
\item \textit{User} \\
\textit{User} merupakan pegawai Direktorat Sumber Daya Manusia Universitas Telkom yang membutuhkan kandidat dengan proses \textit{filtering} untuk mengisi posisi yang kosong atau digantikan.

\item \textit{Admin} \\
\textit{Admin} merupakan pegawai  Direktorat Sumber Daya Manusia Universitas Telkom yang mengelola dan menyediakan data untuk proses \textit{filtering}.
\end{enumerate}

\subsection{Spesifikasi Target Perangkat}

Spesifikasi dari target perangkat untuk mengakses aplikasi \textbf{SiPJabS} adalah sebagai berikut: 

\begin{enumerate}

\item Komputer atau laptop yang terhubung dengan koneksi internet dan dapat membuka web \textit{browser}.

\item	\textit{Smartphone} atau tablet yang terhubung dengan koneksi internet dan dapat membuka web \textit{browser}.
\end{enumerate}

\subsection{Diagram Alir Aplikasi}

Dalam membangun aplikasi \textbf{SiPJabS}, dibutuhkan diagram alir untuk membantu \textit{developer} dan pengguna dalam memahami sistem yang akan dibuat. Berikut merupakan \textit{flowchart} aplikasi:

\begin{figure}
	\centering
	\includegraphics[width=1\textwidth]
	{pics/diagram/flowchart.png}
	\caption{\textit{Flowchart}}
	\label{fig:31}
\end{figure}

Untuk flowchart pertama pengguna mengakses web dan terdapat halaman login, kemudian pengguna memasukkan \textit{username} dan \textit{password}. Jika akun \textit{user} maka akan masuk halaman \textit{user} dan jika akun admin maka akan masuk dalam halaman admin. Pada halaman \textit{user} sendiri terdapat proses pencarian kandidat dengan sistem \textit{filtering} dan \textit{user} dapat mencetak pdf pada halaman data kandidat yang sudah terpilih. Pada halaman admin terdapat beberapa menu guna  pemrosesan data yang disiapkan untuk melakukan proses \textit{filtering}.

\newpage
%-----------------------------------------------------------------------------%
\section{Kebutuhan Pengembangan Sistem}
%-----------------------------------------------------------------------------%

Dalam membangun aplikasi \textbf{SiPJabS}, dibutuhkan beberapa perangkat untuk mengimplementasikannya. Perangkat tersebut dibagi menjadi tiga, yaitu perangkat keras (\textit{hardware}), perangkat lunak (\textit{software}) dan Perangkat \textit{server} / \textit{hosting}. Adapun kebutuhan pengembangan sistem adalah sebagai berikut: 

\subsection{Kubutuhan Perangkat Keras (\textit{Hardware}) }

\textit{Hardware} yang dibutuhkan dalam perancangan dan pembuatan aplikasi \textbf{SiPJabS}
adalah sebagai berikut:


\begin{table}[H]
	\centering
	\caption{Tabel Kebutuhan \textit{Hardware}}
	\begin{tabular}{ | c | l | p{75mm} | }
		\hline
		No. & Perangkat Keras & Spesifikasi \\
		\hline
		\multirow{6}{*}{1} & \multirow{6}{*}{Laptop MSI GL62M} & \textit{Processor} : Intel Core i7-7700HQ \\
		& & \textit{Operating System} : Windows 10 Education \\
		& & RAM : 8 GB \\
		& & \textit{Storage} : 128 GB SSD + 1 TB Hardisk \\
		& & \textit{Graphics Card} :  nVidia Geforce GTX 1050 \\
		& & \textit{Display} : 15.6" FHD, Anti-Glare (1920 x 1080) \\
		
		
		\hline
		
		\multirow{6}{*}{2} & \multirow{6}{*}{Laptop HP Pavilion x360} & \textit{Processor} : Intel Core i3-6100U \\
		& & \textit{Operating System} : Windows 10 Home \\
		& & RAM : 12 GB \\
		& & \textit{Storage} : 500 GB Hardisk\\
		& & \textit{Graphics Card} : Intel HD 520 Graphics \\
		& & \textit{Display} : 13.3" HD, Touch Screen \\
		
		\hline
	\end{tabular}
\end{table}


\newpage
\subsection{Kebutuhkan Perangkat Lunak (\textit{Software})}

\textit{Software} yang dibutuhkan dalam perancangan dan pembuatan aplikasi \textbf{SiPJabS}
adalah sebagai berikut:

\begin{table}[H]
	\centering
	\caption{Tabel Kebutuhan \textit{Software}}
	\begin{tabular}{ | c | l | p{64.5mm} | }
		\hline
		No. & Perangkat Lunak & Kegunaan \\
		\hline
		
		1 & Visual Studio Code & \textit{Text editor} untuk menuliskan \textit{coding} aplikasi \\
				
		\hline
		
		2 & XAMPP & Sebagai \textit{server} yang berdiri sendiri, yang terdiri atas program \textit{Apache} HTTP \textit{Server}, MySQL \textit{database}, dan penerjemah bahasa yang ditulis dengan bahasa pemrograman PHP dan Perl \\
		
		\hline
		
		3 & IBM Rational System Architect  & Sebuah \textit{software} untuk mendesain rancangan sistem aplikasi \\
		
		\hline
		
		4 & Figma & Untuk mendesai \textit{user interface} secara online \\
		
		
		\hline
		
		5 & Microsoft Office Word & Untuk membuat dokumen dan laporan \\
		
		\hline
		
		6 & TexStudio & Untuk membuat laporan dalam latex \\
		
		\hline
		
		7 & Adobe Premier Pro & Editing vidio demo dan vidio promosi \\
		
		\hline
		
		8 & Brave dan Mozila Firefox & Web \textit{browser} \\
		
		\hline
	\end{tabular}
\end{table}

\subsection{Kebutuhan \textit{Hosting}}

\textit{Hosting} yang dibutuhkan dalam perancangan dan pembuatan aplikasi\textbf{ SiPJabS}
adalah sebagai berikut:

\begin{table}[H]
	\centering
	\caption{Tabel Kebutuhan \textit{Hosting}}
	\begin{tabular}{ | c |  p{54mm} | p{64mm} | }
		\hline
		No. & Server & Spesifikasi \\
		\hline
		\multirow{5}{*}{1} & \multirow{5}{*}{Server Indonesia} & \textit{Storage} : 2 GB \\
		& & RAM : 1 GB \\
		& & \textit{Bandwith} : Unlimited \\
		& & \textit{Processor} : 1 Core \\
		& & \textit{Domain} : my.id  \\
	
		\hline
	\end{tabular}
\end{table}

%-----------------------------------------------------------------------------%
\section{Perancangan Model Program}
%-----------------------------------------------------------------------------%
Perancangan model program dalam pembuatan aplikasi \textbf{SiPJabS} antara
lain \textit{Use Case Diagram}, \textit{Use Case Scenario}, \textit{Class Diagram},
\textit{Entity Relationship Diagram} (ERD). Adapun perancangan model program adalah sebagai berikut:

\subsection{Use Case Diagram}

\begin{figure}
	\centering
	\includegraphics[width=1\textwidth]
	{pics/diagram/usecase.png}
	\caption{\textbf{Use Case Diagram}}
	\label{fig:32}
\end{figure}

Lorem ipsum dolor sit amet, consectetuer adipiscing elit. Etiam lobortis facilisis sem.  Nullam nec mi et neque pharetra sollicitudin.  Praesent imperdiet  mi nec ante. Donec ullamcorper, felis non sodales commodo, lectus velit ultrices augue, a dignissim nibh lectus placerat pede. Vivamus nunc nunc, molestie ut, ultricies vel, semper in, velit. Ut porttitor. Praesent in sapien. Lorem ipsum dolor sit amet, consectetuer adipiscing elit. Duis fringilla tristique neque. Sed interdum libero ut metus. Pellentesque placerat.  Nam rutrum augue a leo.  

\subsection{Use Case Skenario}
Berikut merupakan \textit{use case scenario} dalam pembuatan aplikasi \textbf{SiPJabS}:

\begin{enumerate}
	\item Skenario \textit{Login}
	
	Nomor \kern 3.6pc : SP-01 \\
	Nama Use Case : Melakukan \textit{login} \\
	Aktor \kern 4.1 pc : Admin \\
	Tipe \kern 4.6pc : \textit{Primary} \\
	Tujuan \kern 3.6pc : Admin dapat menggunakan aplikasi \\
	Deskripsi \kern 2.5pc : 
	
	\begin{itemize}
		\item Admin menginputkan \textit{username} dan \textit{password}
		\item Sistem akan mencocokkan data
		\item Sistem menampilkan halaman utama aplikasi
	\end{itemize}

	\begin{table}
		\caption{Skenario \textit{Login}}
		\centering
		\begin{tabular}{ | p{60mm} | p{68mm} |}
			\hline 
			\textbf{Aktor} & \textbf{Sistem} \\
			\hline
			
			1.	Menginputkan \textit{username} dan \textit{password} &  \\
			
			\hline
			
			& 2. Mencocokkan data \\
			
			\hline
			
			& 3.	Menampilkan halaman utama aplikasi \\
		
			\hline
			
		\end{tabular}
	\end{table}

\item Skenario Edit \textit{Profile}

Nomor \kern 3.6pc : SP-02 \\
Nama Use Case : Melakukan \textit{edit profile} \\
Aktor \kern 4.1 pc : Admin \\
Tipe \kern 4.6pc : \textit{Primary} \\
Tujuan \kern 3.6pc : Admin dapat melakukan edit pada \textit{profile} \\
Deskripsi \kern 2.5pc : 

\begin{itemize}
	\item Admin menuju ke halaman \textit{profile}
	\item Sistem akan menampilkan halaman \textit{profile}
	\item Admin memilih edit \textit{profile}
	\item Sistem menampilkan \textit{pop-up form edit profile}
	\item Admin menginputkan data
	
\end{itemize}

\begin{table}
	\caption{Skenario Edit \textit{Profile}}
	\centering
	\begin{tabular}{ |  p{59mm} | l |}
		\hline 
		\textbf{Aktor} & \textbf{Sistem} \\
		\hline
		
		1.	Menuju ke halaman \textit{profile} &  \\
		
		\hline
		
		&  2.	Menampilkan halaman \textit{profile} \\
		
		\hline
		
		 3. Memilih edit \textit{profile} & \\
		
		\hline
		
			& 4.	Menampilkan\textit{ pop-up form edit profile} \\
		
		\hline
		
		5.	Menginputkan data  & \\
		\hline
		
		& 6.	Menyimpan data perubahan\\
		\hline
		
	\end{tabular}
\end{table}

\item Skenario \textit{Reset Password}

Nomor \kern 3.6pc : SP-03 \\
Nama Use Case : Melakukan \textit{reset password} \\
Aktor \kern 4.1 pc : Admin \\
Tipe \kern 4.6pc : \textit{Primary} \\
Tujuan \kern 3.6pc : Admin dapat melakukan \textit{reset password} pada \textit{profile} \\
Deskripsi \kern 2.5pc : 

\begin{itemize}
	\item Admin menuju ke halaman \textit{profile}
	\item Sistem akan menampilkan halaman \textit{profile}
	\item Admin memilih\textit{ reset password}
	\item Sistem menampilkan \textit{pop-up reset password}
	\item Admin menginputkan \textit{password} lama dan baru
	\item Sistem menyimpan data perubahan
	
\end{itemize}

\begin{table}
	\caption{Skenario \textit{Reset Password}}
	\centering
	\begin{tabular}{ |  p{61mm} | l |}
		\hline 
		\textbf{Aktor} & \textbf{Sistem} \\
		\hline
		
		1.	Menuju ke halaman \textit{profile} &  \\
		
		\hline
		
		&  2.	Menampilkan halaman \textit{profile} \\
		
		\hline
		
		3. Memilih \textit{reset password} & \\
		
		\hline
		
		& 4.	Menampilkan\textit{ pop-up reset password} \\
		
		\hline
		
		5.	Menginputkan \textit{password} lama dan baru  & \\
		\hline
		
		
		& 6.	Menyimpan data perubahan \\
		\hline
		
	\end{tabular}
\end{table}

\newpage

\item Skenario Tambah \textit{Users}

Nomor \kern 3.6pc : SP-04 \\
Nama Use Case : Menambahkan \textit{users} \\
Aktor \kern 4.1 pc : Admin \\
Tipe \kern 4.6pc : \textit{Primary} \\
Tujuan \kern 3.6pc : Admin dapat menambahkan \textit{users} \\
Deskripsi \kern 2.5pc : 

\begin{itemize}
	\item Admin menuju ke halaman data \textit{users}
	\item Sistem akan menampilkan halaman data \textit{users}
	\item Admin memilih tambah \textit{users}
	\item Sistem menampilkan halaman \textit{form} tambah \textit{users}
	\item Admin menginputkan data
	\item Sistem menyimpan data
	\item Sistem menampilkan \textit{pop-up} tanda berhasil menambahkan \textit{users}
	
\end{itemize}

\begin{table}
	\caption{Skenario Tambah \textit{Users}}
	\centering
	\begin{tabular}{ | l | p{72mm} |}
		\hline 
		\textbf{Aktor} & \textbf{Sistem} \\
		\hline
		
		1.	Menuju ke halaman data \textit{users} &  \\
		
		\hline
		
		&  2. Menampilkan halaman data \textit{users} \\
		
		\hline
		
		3. Memilih tambah \textit{users} & \\
		
		\hline
		
		& 4. Menampilkan halaman \textit{form} tambah \textit{users} \\
		
		\hline
		
		5.	Menginputkan data  & \\
		\hline
		
		
		& 6.	Menyimpan data \\
		\hline
		
		& 7. Menampilkan \textit{pop-up} tanda berhasil menambahkan \textit{users} \\
		\hline
		
	\end{tabular}
\end{table}

\newpage
\item Skenario Edit \textit{Users}

Nomor \kern 3.6pc : SP-05 \\
Nama Use Case : Melakukan edit data \textit{users} \\
Aktor \kern 4.1 pc : Admin \\
Tipe \kern 4.6pc : \textit{Primary} \\
Tujuan \kern 3.6pc : Admin dapat mengedit data \textit{users} \\
Deskripsi \kern 2.5pc : 

\begin{itemize}
	\item Admin menuju ke halaman data \textit{users}
	\item Sistem akan menampilkan halaman data \textit{users}
	\item Admin memilih edit pada salah satu \textit{users}
	\item Sistem menampilkan \textit{pop-up form edit users}
	\item Admin menginputkan data
	\item Sistem menyimpan data perubahan
	\item Sistem menampilkan \textit{pop-up} tanda berhasil mengedit data
	
\end{itemize}

\begin{table}
	\caption{Skenario Edit\textit{ Users}}
	\centering
	\begin{tabular}{ | l | p{65mm} |}
		\hline 
		\textbf{Aktor} & \textbf{Sistem} \\
		\hline
		
		1.	Menuju ke halaman data \textit{users} &  \\
		
		\hline
		
		&  2.	Menampilkan halaman data \textit{users} \\
		
		\hline
		
		3. Memilih \textit{edit} pada salah satu \textit{users} & \\
		
		\hline
		
		& 4.	Menampilkan \textit{pop-up form edit users} \\
		
		\hline
		
		5.	Menginputkan data  & \\
		\hline
		
		& 7.	Menyimpan data \\
		\hline
		
		& 8.	Menampilkan \textit{pop-up} tanda berhasil edit \textit{users} \\
		\hline
		
	\end{tabular}
\end{table}

\newpage

\item Skenario \textit{Delete Users}

Nomor \kern 3.6pc : SP-06 \\
Nama Use Case : Melakukan \textit{delete} data \textit{users} \\
Aktor \kern 4.1 pc : Admin \\
Tipe \kern 4.6pc : \textit{Primary} \\
Tujuan \kern 3.6pc : Admin dapat \textit{delete} data \textit{users} \\
Deskripsi \kern 2.5pc : 

\begin{itemize}
	\item Admin menuju ke halaman data \textit{users}
	\item Sistem akan menampilkan halaman data \textit{users}
	\item Admin memilih \textit{delete} pada salah satu \textit{users}
	\item Sistem menampilkan \textit{pop-up} tanda berhasil hapus data
	
\end{itemize}

\begin{table}
	\caption{Skenario \textit{Delete Users}}
	\centering
	\begin{tabular}{ | l | p{61mm} |}
		\hline 
		\textbf{Aktor} & \textbf{Sistem} \\
		\hline
		
		1.	Menuju ke halaman data \textit{users} &  \\
		
		\hline
		
		&  2.	Menampilkan halaman data \textit{users} \\
		
		\hline
		
		3. Memilih \textit{delete} pada salah satu \textit{users} & \\
		
		\hline
		
		& 4.	Menampilkan \textit{pop-up} tanda berhasil \textit{delete} data \\
		
		\hline
		
	\end{tabular}
\end{table}

\item Skenario \textit{View} Pegawai

Nomor \kern 3.6pc : SP-07 \\
Nama Use Case : Melakukan \textit{view} data pegawai \\
Aktor \kern 4.1 pc : Admin \\
Tipe \kern 4.6pc : \textit{Primary} \\
Tujuan \kern 3.6pc : Admin dapat melihat data pegawai \\
Deskripsi \kern 2.5pc : 

\begin{itemize}
	\item Admin menuju ke halaman data pegawai
	\item Sistem akan menampilkan halaman data pegawai
	\item Admin memilih \textit{view} pada salah satu pegawai
	\item Sistem menampilkan \textit{pop-up} detail pegawai
	\item Admin dapat melihat data detail pegawai
	
\end{itemize}

\begin{table}
	\caption{Skenario \textit{View} Pegawai}
	\centering
	\begin{tabular}{ | p{60mm} | p{68mm} |}
		\hline 
		\textbf{Aktor} & \textbf{Sistem} \\
		\hline
		
		1.	Menuju ke halaman data pegawai &  \\
		
		\hline
		
		&  2.	Menampilkan halaman data pegawai \\
		
		\hline
		
		3. Memilih \textit{view} pada salah satu data pegawai & \\
		
		\hline
		
		& 4.	Menampilkan \textit{pop-up} detail pegawai \\
		
		\hline
		
		5.	Melihat data detail pegawai  & \\
		\hline
		
		
	\end{tabular}
\end{table}

\item Skenario Tambah Pegawai

Nomor \kern 3.6pc : SP-08 \\
Nama Use Case : Melakukan tambah data pegawai \\
Aktor \kern 4.1 pc : Admin \\
Tipe \kern 4.6pc : \textit{Primary} \\
Tujuan \kern 3.6pc : Admin dapat menambahkan data pegawai \\
Deskripsi \kern 2.5pc : 

\begin{itemize}
	\item Admin menuju ke halaman data pegawai
	\item Sistem akan menampilkan halaman data pegawai
	\item Admin memilih tambah data pegawai
	\item Sistem menampilkan \textit{form} tambah pegawai
	\item Admin menginputkan data
	\item Sistem menyimpan data
	\item Sistem menampilkan \textit{pop-up} berhasil menambahkan data
	
\end{itemize}

\begin{table}
	\caption{Skenario Tambah Pegawai}
	\centering
	\begin{tabular}{ | l | p{66.5mm} |}
		\hline 
		\textbf{Aktor} & \textbf{Sistem} \\
		\hline
		
		1.	Menuju ke halaman data pegawai &  \\
		
		\hline
		
		&  2.	Menampilkan halaman data pegawai \\
		
		\hline
		
		3. Memilih tambah data pegawai & \\
		
		\hline
		
		& 4.	Menampilkan \textit{form} tambah pegawai \\
		
		\hline
		
		5. Menginputkan data & \\
		\hline
		
		& 6.	Menyimpan data \\
		
		\hline
		
		& 7.	Menampilkan \textit{pop-up} berhasil menambahkan data \\
		
		\hline
		
		
	\end{tabular}
\end{table}

\item Skenario Edit Pegawai

Nomor \kern 3.6pc : SP-09 \\
Nama Use Case : Melakukan edit data pegawai \\
Aktor \kern 4.1 pc : Admin\\
Tipe \kern 4.6pc : \textit{Primary} \\
Tujuan \kern 3.6pc : Admin dapat mengedit data pegawai \\
Deskripsi \kern 2.5pc : 

\begin{itemize}
	\item Admin menuju ke halaman data pegawai
	\item Sistem akan menampilkan halaman data pegawai
	\item Admin memilih edit pada salah satu pegawai
	\item Sistem menampilkan \textit{form} edit data pegawai
	\item Admin menginputkan data
	\item Sistem menyimpan data perubahan
	\item Sistem menampilkan\textit{ pop-up} berhasil merubah data
	
\end{itemize}

\begin{table}
	\caption{Skenario Edit Pegawai}
	\centering
	\begin{tabular}{ | p{72mm} | p{56mm} |}
		\hline 
		\textbf{Aktor} & \textbf{Sistem} \\
		\hline
		
		1.	Menuju ke halaman data pegawai &  \\
		
		\hline
		
		&  2.	Menampilkan halaman data pegawai \\
		
		\hline
		
		3. Memilih edit pada salah satu pegawai & \\
		
		\hline
		
		& 4.	Menampilkan \textit{form} edit data pegawai \\
		
		\hline
		
		5.	Menginpputkan data  & \\
		\hline
		
			& 6.	Menyimpan data perubahan \\
		
		\hline
		
		& 7.	Menampilkan \textit{pop-up} berhasil merubah data \\
		
		\hline
		
		
	\end{tabular}
\end{table}

\newpage

\item Skenario \textit{View} Kandidat

Nomor \kern 3.6pc : SP-10 \\
Nama Use Case : Melakukan \textit{view} data kandidat \\
Aktor \kern 4.1 pc : Admin \\
Tipe \kern 4.6pc : \textit{Primary} \\
Tujuan \kern 3.6pc : Admin dapat melihat data kandidat \\
Deskripsi \kern 2.5pc : 

\begin{itemize}
	\item Admin menuju ke halaman data kandidat
	\item Sistem akan menampilkan halaman data kandidat
	\item Admin memilih \textit{view} pada salah satu kandidat
	\item Sistem menampilkan \textit{pop-up} detail data kandidat
	\item Admin dapat melihat data detail kandidat
	
\end{itemize}

\begin{table}
	\caption{Skenario \textit{View} Kandidat}
	\centering
	\begin{tabular}{ |  p{73mm} | p{55mm} |}
		\hline 
		\textbf{Aktor} & \textbf{Sistem} \\
		\hline
		
		1.	Menuju ke halaman data kandidat &  \\
		
		\hline
		
		&  2.	Menampilkan halaman data kandidat \\
		
		\hline
		
		3. Memilih \textit{view} pada salah satu kandidat & \\
		
		\hline
		
		& 4.	Menampilkan \textit{pop-up} detail data kandidat \\
		
		\hline
		
		5. Melihat data detail kandidat  & \\
		\hline
		
		
	\end{tabular}
\end{table}

\item Skenario \textit{Print Form} Kandidat

Nomor \kern 3.6pc : SP-11 \\
Nama Use Case : Melakukan \textit{print form} data kandidat \\
Aktor \kern 4.1 pc : Admin \\
Tipe \kern 4.6pc : \textit{Primary} \\
Tujuan \kern 3.6pc : Admin dapat mencetak \textit{form} data kandidat \\
Deskripsi \kern 2.5pc : 

\begin{itemize}
	\item Admin menuju ke halaman data kandidat
	\item Sistem akan menampilkan halaman data kandidat
	\item Admin memilih \textit{print} pdf pada salah satu data kandidat
	
\end{itemize}

\begin{table}
	\caption{Skenario \textit{Print Form} Kandidat}
	\centering
	\begin{tabular}{ | p{61.5mm} | p{67mm} |}
		\hline 
		\textbf{Aktor} & \textbf{Sistem} \\
		\hline
		
		1.	Menuju ke halaman data kandidat &  \\
		
		\hline
		
		&  2.	Menampilkan halaman data kandidat \\
		
		\hline
		
		3. Memilih \textit{print} pdf pada salah satu data kandidat & \\
		
		\hline
	
		
	\end{tabular}
\end{table}

\item Skenario Tambah Unit Kerja

Nomor \kern 3.6pc : SP-12 \\
Nama Use Case : Menambahkan data unit kerja \\
Aktor \kern 4.1 pc : Admin \\
Tipe \kern 4.6pc : \textit{Primary} \\
Tujuan \kern 3.6pc : Admin dapat menambahkan data unit kerja \\
Deskripsi \kern 2.5pc : 

\begin{itemize}
	\item Admin menuju ke halaman data unit kerja
	\item Sistem akan menampilkan halaman data unit kerja
	\item Admin memilih tambah unit kerja
	\item Sistem menampilkan \textit{pop-up} form tambah unit kerja
	\item Admin menginputkan data
	\item Sistem menyimpan data
	\item Sistem menampilkan\textit{ pop-up} tanda data berhasil ditambahkan
	
\end{itemize}

\begin{table}
	\caption{Skenario Tambah Unit Kerja}
	\centering
	\begin{tabular}{ | l | p{65mm} |}
		\hline 
		\textbf{Aktor} & \textbf{Sistem} \\
		\hline
		
		1.	Menuju ke halaman data unit kerja &  \\
		
		\hline
		
		&  2.	Menampilkan halaman data unit kerja \\
		
		\hline
		
		3. Memilih tambah unit kerja & \\
		
		\hline
		
		& 4.	Menampilkan \textit{pop-up} form tambah unit kerja \\
		
		\hline
		
		5.	Menginputkan data  & \\
		\hline
		
		& 6.	Menyimpan data \\
		\hline
		
		& 7.	Menampilkan \textit{pop-up} tanda berhasil menambahkan data \\
		\hline
		
	\end{tabular}
\end{table}

\item Skenario Edit Unit Kerja

Nomor \kern 3.6pc : SP-13 \\
Nama Use Case : Melakuan edit data unit kerja \\
Aktor \kern 4.1 pc : Admin \\
Tipe \kern 4.6pc : \textit{Primary} \\
Tujuan \kern 3.6pc : Admin dapat mengedit data unit kerja \\
Deskripsi \kern 2.5pc : 

\begin{itemize}
	\item Admin menuju ke halaman data unit kerja
	\item Sistem akan menampilkan halaman data unit kerja
	\item Admin memilih edit pada suatu data unit kerja
	\item Sistem menampilkan \textit{pop-up form}  edit unit kerja
	\item Admin menginputkan data
	\item Sistem menyimpan data
	\item Sistem menampilkan \textit{pop-up} tanda berhasil edit data
	
\end{itemize}

\begin{table}
	\caption{Skenario Edit Unit Kerja}
	\centering
	\begin{tabular}{ | p{58mm} | p{70mm} |}
		\hline 
		\textbf{Aktor} & \textbf{Sistem} \\
		\hline
		
		1.	Menuju ke halaman data unit kerja &  \\
		
		\hline
		
		&  2.	Menampilkan halaman data unit kerja \\
		
		\hline
		
		3. Memilih edit pada suatu data unit kerja & \\
		
		\hline
		
		& 4.	Menampilkan \textit{pop-up form}  edit unit kerja \\
		
		\hline
		
		5.	Menginputkan data  & \\
		\hline
		
		& 6.	Menyimpan data \\
		\hline
		
		& 7.	Menampilkan \textit{pop-up }tanda berhasil edit data \\
		\hline
		
	\end{tabular}
\end{table}

\newpage
\item Skenario \textit{View} Unit Kerja

Nomor \kern 3.6pc : SP-14 \\
Nama Use Case : Melihat pegawai yang memiliki data unit kerja \\
Aktor \kern 4.1 pc : Admin \\
Tipe \kern 4.6pc : \textit{Primary} \\
Tujuan \kern 3.6pc : Admin dapat melihat data pegawai sesuai unit kerja \\
Deskripsi \kern 2.5pc : 

\begin{itemize}
	\item Admin menuju ke halaman data unit kerja
	\item Sistem akan menampilkan halaman data unit kerja
	\item Admin memilih \textit{view} pada suatu data unit kerja
	\item Sistem menampilkan\textit{ pop-up} data pegawai yang sesuai unit kerja
	\item Admin dapat melihat data pegawai yang sesuai unit kerja
	
\end{itemize}

\begin{table}
	\caption{Skenario \textit{View} Unit Kerja}
	\centering
	\begin{tabular}{ | p{63mm} | p{65mm} |}
		\hline 
		\textbf{Aktor} & \textbf{Sistem} \\
		\hline
		
		1.	Menuju ke halaman data unit kerja &  \\
		
		\hline
		
		&  2. Menampilkan halaman data unit kerja \\
		
		\hline
		
		3. Memilih \textit{view} pada suatu data unit kerja & \\
		
		\hline
		
		& 4. Menampilkan \textit{pop-up} data pegawai yang sesuai unit kerja \\
		\hline
		
		5. Melihat data pegawai yang sesuai unit kerja & \\
		
		\hline
		
		
	\end{tabular}
\end{table}

\item Skenario Tambah Jabatan

Nomor \kern 3.6pc : SP-15 \\
Nama Use Case : Menambahkan data jabatan \\
Aktor \kern 4.1 pc : Admin \\
Tipe \kern 4.6pc : \textit{Primary} \\
Tujuan \kern 3.6pc : Admin dapat menambahkan data jabatan \\
Deskripsi \kern 2.5pc : 

\begin{itemize}
	\item Admin menuju ke halaman data jabatan
	\item Sistem akan menampilkan halaman data jabatan
	\item Admin memilih tambah jabatan
	\item Sistem menampilkan \textit{pop-up} tambah jabatan
	\item Admin menginputkan data
	\item Sistem menyimpan data
	\item Sistem menampilkan \textit{pop-up} tanda berhasil ditambahkan
	
\end{itemize}

\begin{table}
	\caption{Skenario Tambah Jabatan}
	\centering
	\begin{tabular}{ | l | p{68.5mm} |}
		\hline 
		\textbf{Aktor} & \textbf{Sistem} \\
		\hline
		
		1.	Menuju ke halaman data jabatan &  \\
		
		\hline
		
		&  2.	Menampilkan halaman data jabatan \\
		
		\hline
		
		3. Memilih tambah jabatan & \\
		
		\hline
		
		& 4.	Menampilkan \textit{pop-up} tambah jabatan \\
		
		\hline
		
		5.	Menginputkan data  & \\
		\hline
		
		& 6.	Menyimpan data \\
		\hline
		
		& 7.	Menampilkan \textit{pop-up} tanda berhasil menambahkan data \\
		\hline
		
	\end{tabular}
\end{table}

\item Skenario Edit Jabatan

Nomor \kern 3.6pc : SP-16 \\
Nama Use Case : Melakukan edit data jabatan \\
Aktor \kern 4.1 pc : Admin \\
Tipe \kern 4.6pc : \textit{Primary} \\
Tujuan \kern 3.6pc : Admin dapat mengedit data jabatan \\
Deskripsi \kern 2.5pc : 

\begin{itemize}
	\item Admin menuju ke halaman data jabatan
	\item Sistem akan menampilkan halaman data jabatan
	\item Admin memilih edit pada suatu data jabatan
	\item Sistem menampilkan \textit{pop-up form} edit jabatan
	\item Admin menginputkan data
	\item Sistem menyimpan data perubahan
	\item Sistem menampilkan \textit{pop-up} tanda berhasil edit data
	
\end{itemize}

\begin{table}
	\caption{Skenario Edit Jabatan}
	\centering
	\begin{tabular}{ | p{58mm} | p{70mm} |}
		\hline 
		\textbf{Aktor} & \textbf{Sistem} \\
		\hline
		
		1.	Menuju ke halaman data jabatan &  \\
		
		\hline
		
		&  2.	Menampilkan halaman data jabatan \\
		
		\hline
		
		3. Memilih edit pada suatu jabatan & \\
		
		\hline
		
		& 4.	Menampilkan\textit{ pop-up form} edit jabatan \\
		
		\hline
		
		5.	Menginputkan data  & \\
		\hline
		
		& 6.	Menyimpan data perubahan \\
		\hline
		
		& 7.	Menampilkan \textit{pop-up} tanda berhasil edit data \\
		\hline
		
	\end{tabular}
\end{table}

\item Skenario \textit{View} Jabatan

Nomor \kern 3.6pc : SP-17 \\
Nama Use Case : Melihat pegawai yang memiliki data jabatan \\
Aktor \kern 4.1 pc : Admin \\
Tipe \kern 4.6pc : \textit{Primary} \\
Tujuan \kern 3.6pc : Admin dapat melihat data pegawai sesuai jabatan \\
Deskripsi \kern 2.5pc : 

\begin{itemize}
	\item Admin menuju ke halaman data jabatan
	\item Sistem akan menampilkan halaman data jabatan
	\item Admin memilih \textit{view} pada suatu data jabatan
	\item Sistem menampilkan \textit{pop-up} data pegawai yang sesuai jabatan
	
\end{itemize}

\begin{table}
	\caption{Skenario \textit{View} Jabatan}
	\centering
	\begin{tabular}{ | p{59mm} | p{69mm}|}
		\hline 
		\textbf{Aktor} & \textbf{Sistem} \\
		\hline
		
		1.	Menuju ke halaman data jabatan &  \\
		
		\hline
		
		&  2.	Menampilkan halaman data jabatan \\
		
		\hline
		
		3. Memilih \textit{view} pada suatu data jabatan & \\
		
		\hline
		
		& 4.	Menampilkan \textit{pop-up} data pegawai yang sesuai jabatan\\
		\hline
		
	\end{tabular}
\end{table}
\newpage

\item Skenario Tambah Unit Bagian

Nomor \kern 3.6pc : SP-18 \\
Nama Use Case : Menambahkan data unit bagian \\
Aktor \kern 4.1 pc : Admin \\
Tipe \kern 4.6pc : \textit{Primary} \\
Tujuan \kern 3.6pc : Admin dapat menambahkan data unit bagian \\
Deskripsi \kern 2.5pc : 

\begin{itemize}
	\item Admin menuju ke halaman data unit bagian
	\item Sistem akan menampilkan halaman data unit bagian
	\item Admin memilih tambah unit bagian
	\item Sistem menampilkan \textit{pop-up} form tambah unit bagian
	\item Admin menginputkan data
	\item Sistem menyimpan data
	\item Sistem menampilkan \textit{pop-up} tanda berhasil ditambahkan
	
\end{itemize}

\begin{table}
	\caption{Skenario Tambah Unit Bagian}
	\centering
	\begin{tabular}{ | p{55mm} | p{73mm} |}
		\hline 
		\textbf{Aktor} & \textbf{Sistem} \\
		\hline
		
		1.	Menuju ke halaman data unit bagian &  \\
		
		\hline
		
		&  2.	Menampilkan halaman data unit bagian \\
		
		\hline
		
		3. Memilih tambah unit bagian & \\
		
		\hline
		
		& 4.	Menampilkan\textit{ pop-up} form tambah unit bagian \\
		
		\hline
		
		5.	Menginputkan data  & \\
		\hline
		
		& 6.	Menyimpan data \\
		\hline
		
		& 7.	Menampilkan \textit{pop-up} tanda berhasil menambahkan data \\
		\hline
		
	\end{tabular}
\end{table}

\item Skenario Edit Unit Bagian

Nomor \kern 3.6pc : SP-19 \\
Nama Use Case : Melakukan edit data unit bagian \\
Aktor \kern 4.1 pc : Admin \\
Tipe \kern 4.6pc : \textit{Primary} \\
Tujuan \kern 3.6pc : Admin dapat mengedit data unit bagian \\
Deskripsi \kern 2.5pc : 

\begin{itemize}
	\item Admin menuju ke halaman data unit bagian
	\item Sistem akan menampilkan halaman data unit bagian
	\item Admin memilih edit pada suatu data unit bagian
	\item Sistem menampilkan \textit{pop-up form} edit unit bagian
	\item Admin menginputkan data
	\item Sistem menyimpan data perubahan
	\item Sistem menampilkan \textit{pop-up} tanda berhasil di edit
	
\end{itemize}

\begin{table}
	\caption{Skenario Edit Unit Bagian}
	\centering
	\begin{tabular}{ | p{55mm} | p{73mm} |}
		\hline 
		\textbf{Aktor} & \textbf{Sistem} \\
		\hline
		
		1.	Menuju ke halaman data unit bagian &  \\
		
		\hline
		
		&  2. Menampilkan halaman data unit bagian \\
		
		\hline
		
		3. Memilih edit pada suatu data unit bagian & \\
		
		\hline
		
		& 4.	Menampilkan \textit{pop-up form} edit unit bagian \\
		
		\hline
		
		5.	Menginputkan data  & \\
		\hline
		
		& 6.	Menyimpan data perubahan \\
		\hline
		
		& 7.	Menampilkan \textit{pop-up} tanda berhasil di edit \\
		\hline
		
	\end{tabular}
\end{table}

\item Skenario \textit{View} Unit Bagian

Nomor \kern 3.6pc : SP-20 \\
Nama Use Case : Melihat pegawai yang memiliki data unit bagian \\
Aktor \kern 4.1 pc : Admin \\
Tipe \kern 4.6pc : \textit{Primary} \\
Tujuan \kern 3.6pc : Admin dapat melihat data pegawai sesuai unit bagian \\
Deskripsi \kern 2.5pc : 

\begin{itemize}
	\item Admin menuju ke halaman data unit bagian
	\item Sistem akan menampilkan halaman data unit bagian
	\item Admin memilih \textit{view} pada suatu data unit bagian
	\item Sistem menampilkan \textit{pop-up} data pegawai yang sesuai unit bagian
	
\end{itemize}

\begin{table}
	\caption{Skenario \textit{View} Unit Bagian}
	\centering
	\begin{tabular}{ | p{58mm} | p{70mm} |}
		\hline 
		\textbf{Aktor} & \textbf{Sistem} \\
		\hline
		
		1.	Menuju ke halaman data unit bagian &  \\
		
		\hline
		
		&  2.	Menampilkan halaman data unit bagian \\
		
		\hline
		
		3. Memilih \textit{view} pada suatu data unit bagian & \\
		
		\hline
		
		& 4.	Menampilkan \textit{pop-up} data pegawai yang sesuai unit bagian \\
		\hline
		
	\end{tabular}
\end{table}

\item Skenario Tambah Jabatan Struktural

Nomor \kern 3.6pc : SP-21 \\
Nama Use Case : Menambahkan data jabatan struktural \\
Aktor \kern 4.1 pc : Admin \\
Tipe \kern 4.6pc : \textit{Primary} \\
Tujuan \kern 3.6pc : Admin dapat menambahkan data jabatan struktural\\
Deskripsi \kern 2.5pc : 

\begin{itemize}
	\item Admin menuju ke halaman data jabatan struktural
	\item Sistem akan menampilkan halaman data jabatan struktural
	\item Admin memilih tambah jabatan struktural
	\item Sistem menampilkan \textit{pop-up form} tambah jabatan struktural
	\item Admin menginputkan data
	\item Sistem menyimpan data
	\item Sistem menampilkan \textit{pop-up} tanda berhasil menambahkan data
	
\end{itemize}

\begin{table}
	\caption{Skenario Tambah Jabatan Struktural}
	\centering
	\begin{tabular}{ | p{60mm} | p{68mm} |}
		\hline 
		\textbf{Aktor} & \textbf{Sistem} \\
		\hline
		
		1.	Menuju ke halaman data jabatan struktural &  \\
		
		\hline
		
		&  2.	Menampilkan halaman data jabatan struktural\\
		
		\hline
		
		3. Memilih tambah data jabatan struktural& \\
		
		\hline
		
		& 4.	Menampilkan \textit{pop-up form} tambah jabatan struktural\\
		
		\hline
		
		5.	Menginputkan data  & \\
		\hline
		
		& 6.	Menyimpan data \\
		\hline
		
		& 7.	Menampilkan \textit{pop-up} tanda berhasil menambahkan data \\
		\hline
		
	\end{tabular}
\end{table}

\item Skenario Edit Jabatan Struktural

Nomor \kern 3.6pc : SP-22 \\
Nama Use Case : Melakukan edit data jabatan struktural \\
Aktor \kern 4.1 pc : Admin \\
Tipe \kern 4.6pc : \textit{Primary} \\
Tujuan \kern 3.6pc : Admin dapat mengedit data jabatan struktural\\
Deskripsi \kern 2.5pc : 

\begin{itemize}
	\item Admin menuju ke halaman data jabatan struktural
	\item Sistem akan menampilkan halaman data jabatan struktural
	\item Admin memilih edit pada suatu data jabatan struktural
	\item Sistem menampilkan \textit{pop-up form} edit jabatan struktural
	\item Admin menginputkan data
	\item Sistem menyimpan data perubahan
	\item Sistem menampilkan \textit{pop-up} tanda berhasil diedit
	
\end{itemize}

\begin{table}
	\caption{Skenario Edit Jabatan Struktural}
	\centering
	\begin{tabular}{ |p{75mm} | p{53mm} |}
		\hline 
		\textbf{Aktor} & \textbf{Sistem} \\
		\hline
		
		1.	Menuju halaman data jabatan struktural &  \\
		
		\hline
		
		&  2.	Menampilkan halaman data jabatan struktural \\
		
		\hline
		
		3. Memilih edit pada suatu jabatan struktural & \\
		
		\hline
		
		& 4.	Menampilkan \textit{pop-up form} edit jabatan struktural\\
		
		\hline
		
		5.	Menginputkan data  & \\
		\hline
		
		& 6.	Menyimpan data perubahan \\
		\hline
		
		& 7.	Menampilkan \textit{pop-up} tanda berhasil edit data \\
		\hline
		
	\end{tabular}
\end{table}

\item Skenario \textit{View} Jabatan Struktural

Nomor \kern 3.6pc : SP-23\\
Nama Use Case : Melihat pegawai yang memiliki data jabatan struktural \\
Aktor \kern 4.1 pc : Admin \\
Tipe \kern 4.6pc : \textit{Primary} \\
Tujuan \kern 3.6pc : Admin dapat melihat data pegawai sesuai jabatan struktural\\
Deskripsi \kern 2.5pc : 

\begin{itemize}
	\item Admin menuju ke halaman data jabatan struktural
	\item Sistem akan menampilkan halaman data jabatan struktural
	\item Admin memilih \textit{view} pada suatu data jabatan struktural
	\item Sistem menampilkan \textit{pop-up} data pegawai yang sesuai jabatan struktural
	
\end{itemize}

\begin{table}
	\caption{Skenario \textit{View} Jabatan Struktural}
	\centering
	\begin{tabular}{ | p{75.5mm} | p{53mm}|}
		\hline 
		\textbf{Aktor} & \textbf{Sistem} \\
		\hline
		
		1.	Menuju ke halaman data jabatan struktural &  \\
		
		\hline
		
		&  2.	Menampilkan halaman data jabatan struktural\\
		
		\hline
		
		3. Memilih \textit{view} pada suatu data jabatan struktural& \\
		
		\hline
		
		& 4.	Menampilkan \textit{pop-up} data pegawai yang sesuai jabatan struktural \\
		\hline
		
	\end{tabular}
\end{table}

\item Skenario Tambah Pendidikan

Nomor \kern 3.6pc : SP-24 \\
Nama Use Case : Menambahkan data pendidikn \\
Aktor \kern 4.1 pc : Admin \\
Tipe \kern 4.6pc : \textit{Primary} \\
Tujuan \kern 3.6pc : Admin dapat menambahkan data pendidikan \\
Deskripsi \kern 2.5pc : 

\begin{itemize}
	\item Admin menuju ke halaman data pendidikan
	\item Sistem akan menampilkan halaman data pendidikan
	\item Admin memilih tambah pendidikan
	\item Sistem menampilkan \textit{pop-up form} tambah pendidikan
	\item Admin menginputkan data
	\item Sistem menyimpan data
	\item Sistem menampilkan \textit{pop-up} tanda berhasil ditambahkan
	
\end{itemize}

\begin{table}
	\caption{Skenario Tambah Pendidikan}
	\centering
	\begin{tabular}{ | l | p{62mm} |}
		\hline 
		\textbf{Aktor} & \textbf{Sistem} \\
		\hline
		
		1.	Menuju ke halaman data pendidikan &  \\
		
		\hline
		
		&  2.	Menampilkan halaman data pendidikan \\
		
		\hline
		
		3. Memilih tambah pendidikan & \\
		
		\hline
		
		& 4.	Menampilkan \textit{pop-up form} tambah pendidikan \\
		
		\hline
		
		5.	Menginputkan data  & \\
		\hline
		
		& 6.	Menyimpan data \\
		\hline
		
		& 7.	Menampilkan \textit{pop-up} tanda berhasil menambahkan data \\
		\hline
		
	\end{tabular}
\end{table}

\item Skenario Edit Pendidikan

Nomor \kern 3.6pc : SP-25 \\
Nama Use Case : Melakukan edit data pendidikan \\
Aktor \kern 4.1 pc : Admin \\
Tipe \kern 4.6pc : \textit{Primary} \\
Tujuan \kern 3.6pc : Admin dapat mengedit data pendidikan \\
Deskripsi \kern 2.5pc : 

\begin{itemize}
	\item Admin menuju ke halaman data pendidikan
	\item Sistem akan menampilkan halaman data pendidikan
	\item Admin memilih edit pada suatu data pendidikan
	\item Sistem menampilkan \textit{pop-up form} edit pendidikan
	\item Admin menginputkan data
	\item Sistem menyimpan data perubahan
	\item Sistem menampilkan\textit{ pop-up} tanda berhasil di edit
	
\end{itemize}

\begin{table}
	\caption{Skenario Edit Pendidikan}
	\centering
	\begin{tabular}{ | p{67mm} | p{61mm} |}
		\hline 
		\textbf{Aktor} & \textbf{Sistem} \\
		\hline
		
		1.	Menuju ke halaman data pendidikan &  \\
		
		\hline
		
		&  2.	Menampilkan halaman data pendidikan \\
		
		\hline
		
		3. Memilih edit pada suatu pendidikan & \\
		
		\hline
		
		& 4.	Menampilkan \textit{pop-up form} edit pendiaikan \\
		
		\hline
		
		5.	Menginputkan data  & \\
		\hline
		
		& 6.	Menyimpan data perubahan \\
		\hline
		
		& 7.	Menampilkan \textit{pop-up} tanda berhasil edit data \\
		\hline
		
	\end{tabular}
\end{table}

\item Skenario \textit{View} Pendidikan

Nomor \kern 3.6pc : SP-26 \\
Nama Use Case : Melihat pegawai yang memiliki data pendidikan \\
Aktor \kern 4.1 pc : Admin \\
Tipe \kern 4.6pc : \textit{Primary} \\
Tujuan \kern 3.6pc : Admin dapat melihat data pegawai sesuai pendidikan \\
Deskripsi \kern 2.5pc : 

\begin{itemize}
	\item Admin menuju ke halaman data pendidikan
	\item Sistem akan menampilkan halaman data pendidikan
	\item Admin memilih \textit{view} pada suatu data pendidikan
	\item Sistem menampilkan \textit{pop-up} data pegawai sesuai pendidikan 
	
\end{itemize}

\begin{table}
	\caption{Skenario \textit{View} Pendidikan}
	\centering
	\begin{tabular}{ | p{53mm} | p{75mm}|}
		\hline 
		\textbf{Aktor} & \textbf{Sistem} \\
		\hline
		
		1.	Menuju ke halaman data pendidikan &  \\
		
		\hline
		
		&  2.	Menampilkan halaman data pendidikan \\
		
		\hline
		
		3. Memilih \textit{view} pada suatu data pendidikan & \\
		
		\hline
		
		& 4.	Menampilkan \textit{pop-up} data pegawai sesuai pendidikan  \\
		\hline
		
	\end{tabular}
\end{table}

\item Skenario Tambah Skill

Nomor \kern 3.6pc : SP-27 \\
Nama Use Case : Menambahkan data skill \\
Aktor \kern 4.1 pc : Admin \\
Tipe \kern 4.6pc : \textit{Primary} \\
Tujuan \kern 3.6pc : Admin dapat menambahkan data skill \\
Deskripsi \kern 2.5pc : 

\begin{itemize}
	\item Admin menuju ke halaman data skill
	\item Sistem akan menampilkan halaman data skill
	\item Admin memilih tambah skill
	\item Sistem menampilkan \textit{pop-up form} tambah skill
	\item Admin menginputkan data
	\item Sistem menyimpan data
	\item Sistem menampilkan \textit{pop-up} tanda berhasil ditambahkan
	
\end{itemize}

\begin{table}
	\caption{Skenario Tambah Skill}
	\centering
	\begin{tabular}{ | l | p{73.5mm} |}
		\hline 
		\textbf{Aktor} & \textbf{Sistem} \\
		\hline
		
		1.	Menuju ke halaman data skill &  \\
		
		\hline
		
		&  2.	Menampilkan halaman data skill \\
		
		\hline
		
		3. Memilih tambah skill & \\
		
		\hline
		
		& 4.	Menampilkan \textit{pop-up form} tambah skill \\
		
		\hline
		
		5.	Menginputkan data  & \\
		\hline
		
		& 6.	Menyimpan data \\
		\hline
		
		& 7.	Menampilkan \textit{pop-up} tanda berhasil menambahkan data \\
		\hline
		
	\end{tabular}
\end{table}

\item Skenario Edit \textit{Skill}

Nomor \kern 3.6pc : SP-28 \\
Nama Use Case : Melakukan edit data \textit{skill} \\
Aktor \kern 4.1 pc : Admin \\
Tipe \kern 4.6pc : \textit{Primary} \\
Tujuan \kern 3.6pc : Admin dapat mengedit data \textit{skill} \\
Deskripsi \kern 2.5pc : 

\begin{itemize}
	\item Admin menuju ke halaman data \textit{skill}
	\item Sistem akan menampilkan halaman data \textit{skill}
	\item Admin memilih edit pada suatu data \textit{skill}
	\item Sistem menampilkan \textit{pop-up form} edit \textit{skill}
	\item Admin menginputkan data
	\item Sistem menyimpan data perubahan
	\item Sistem menampilkan \textit{pop-up} tanda berhasil diedit
	
\end{itemize}

\begin{table}
	\caption{Skenario Edit \textit{Skill}}
	\centering
	\begin{tabular}{ | l | p{66mm} |}
		\hline 
		\textbf{Aktor} & \textbf{Sistem} \\
		\hline
		
		1.	Menuju ke halaman data \textit{skill} &  \\
		
		\hline
		
		&  2.	Menampilkan halaman data \textit{skill} \\
		
		\hline
		
		3. Memilih edit pada suatu data \textit{skill} & \\
		
		\hline
		
		& 4.	Menampilkan \textit{pop-up form} edit \textit{skill} \\
		
		\hline
		
		5.	Menginputkan data  & \\
		\hline
		
		& 6.	Menyimpan data perubahan \\
		\hline
		
		& 7.	Menampilkan \textit{pop-up} tanda berhasil edit data \\
		\hline
		
	\end{tabular}
\end{table}

\item Skenario \textit{View Skill} 

Nomor \kern 3.6pc : SP-29 \\
Nama Use Case : Melihat pegawai yang memiliki data \textit{skill} \\
Aktor \kern 4.1 pc : Admin \\
Tipe \kern 4.6pc : \textit{Primary} \\
Tujuan \kern 3.6pc : Admin dapat melihat data pegawai yang sesuai \textit{skill} \\
Deskripsi \kern 2.5pc : 
\\
\begin{itemize}
	\item Admin menuju ke halaman data \textit{skill}
	\item Sistem akan menampilkan halaman data \textit{skill}
	\item Admin memilih \textit{view} pada suatu data \textit{skill}
	\item Sistem menampilkan \textit{pop-up} data pegawai yang sesuai \textit{skill}
	
\end{itemize}

\begin{table}
	\caption{Skenario \textit{View Skill}}
	\centering
	\begin{tabular}{ | l | p{65mm}|}
		\hline 
		\textbf{Aktor} & \textbf{Sistem} \\
		\hline
		
		1.	Menuju ke halaman data \textit{skill} &  \\
		
		\hline
		
		&  2.	Menampilkan halaman data \textit{skill} \\
		
		\hline
		
		3. Memilih view pada suatu data \textit{skill} & \\
		
		\hline
		
		& 4.	Menampilkan \textit{pop-up} data pegawai yang sesuai \textit{skill} \\
		\hline
		
	\end{tabular}
\end{table}

\item Skenario Tambah \textit{Personal Quality}

Nomor \kern 3.6pc : SP-30 \\
Nama Use Case : Menambahkan data \textit{personal quality} \\
Aktor \kern 4.1 pc : Admin \\
Tipe \kern 4.6pc : \textit{Primary} \\
Tujuan \kern 3.6pc : Admin dapat menambahkan data \textit{personal quality} \\
Deskripsi \kern 2.5pc : 

\begin{itemize}
	\item Admin menuju ke halaman data \textit{personal quality}
	\item Sistem akan menampilkan halaman data \textit{personal quality}
	\item Admin memilih tambah \textit{personal quality}
	\item Sistem menampilkan\textit{ pop-up form}  tambah \textit{personal quality}
	\item Admin menginputkan data
	\item Sistem menyimpan data
	\item Sistem menampilkan \textit{pop-up} tanda berhasil ditambahkan
	
\end{itemize}

\begin{table}
	\caption{Skenario Tambah \textit{Personal Quality}}
	\centering
	\begin{tabular}{ | l | p{54.5mm} |}
		\hline 
		\textbf{Aktor} & \textbf{Sistem} \\
		\hline
		
		1.	Menuju ke halaman data \textit{personal quality} &  \\
		
		\hline
		
		&  2.	Menampilkan halaman data \textit{personal quality} \\
		
		\hline
		
		3. Memilih tambah \textit{personal quality} & \\
		
		\hline
		
		& 4.	Menampilkan\textit{ pop-up form} tambah \textit{personal quality} \\
		
		\hline
		
		5.	Menginputkan data  & \\
		\hline
		
		& 6.	Menyimpan data \\
		\hline
		
		& 7.	Menampilkan pop-up tanda berhasil menambahkan data \\
		\hline
		
	\end{tabular}
\end{table}

\item Skenario Edit\textit{ Personal Quality}

Nomor \kern 3.6pc : SP-31 \\
Nama Use Case : Melakukan edit data \textit{personal quality} \\
Aktor \kern 4.1 pc : Admin \\
Tipe \kern 4.6pc : \textit{Primary} \\
Tujuan \kern 3.6pc : Admin dapat mengedit data \textit{personal quality} \\
Deskripsi \kern 2.5pc : 

\begin{itemize}
	\item Admin menuju ke halaman data \textit{personal quality}
	\item Sistem akan menampilkan halaman data\textit{ personal quality}
	\item Admin memilih edit pada suatu data \textit{personal quality}
	\item Sistem menampilkan \textit{pop-up form} edit \textit{personal quality}
	\item Admin menginputkan data
	\item Sistem menyimpan data perubahan
	\item Sistem menampilkan\textit{ pop-up }tanda berhasil diedit
	
\end{itemize}

\begin{table}
	\caption{Skenario Edit \textit{Personal Quality}}
	\centering
	\begin{tabular}{ | l | p{54.5mm} |}
		\hline 
		\textbf{Aktor} & \textbf{Sistem} \\
		\hline
		
		1.	Menuju ke halaman data \textit{personal quality} &  \\
		
		\hline
		
		&  2.	Menampilkan halaman data \textit{personal quality} \\
		
		\hline
		
		3. Memilih edit pada suatu \textit{personal quality} & \\
		
		\hline
		
		& 4.	Menampilkan\textit{ pop-up form} edit \textit{personal quality} \\
		
		\hline
		
		5.	Menginputkan data  & \\
		\hline
		
		& 6.	Menyimpan data perubahan \\
		\hline
		
		& 7.	Menampilkan\textit{ pop-up }tanda berhasil edit data \\
		\hline
		
	\end{tabular}
\end{table}

\item Skenario \textit{View Personal Quality}

Nomor \kern 3.6pc : SP-32 \\
Nama Use Case : Melihat pegawai yang memiliki data \textit{personal quality} \\
Aktor \kern 4.1 pc : Admin \\
Tipe \kern 4.6pc : \textit{Primary} \\
Tujuan \kern 3.6pc : Admin dapat melihat data pegawai sesuai \textit{personal quality} \\
Deskripsi \kern 2.5pc : 

\begin{itemize}
	\item Admin menuju ke halaman data \textit{personal quality}
	\item Sistem akan menampilkan halaman data \textit{personal quality}
	\item Admin memilih \textit{view} pada suatu data \textit{personal quality}
	\item Sistem menampilkan \textit{pop-up} data pegawai yang sesuai \textit{personal quality}
	
\end{itemize}

\begin{table}
	\caption{Skenario \textit{View Personal Quality}}
	\centering
	\begin{tabular}{ | l | p{52.5mm}|}
		\hline 
		\textbf{Aktor} & \textbf{Sistem} \\
		\hline
		
		1.	Menuju ke halaman data \textit{personal quality} &  \\
		
		\hline
		
		&  2.	Menampilkan halaman data \textit{personal quality} \\
		
		\hline
		
		3. Memilih \textit{view} pada suatu\textit{ personal quality} & \\
		
		\hline
		
		& 4.	Menampilkan \textit{pop-up} data pegawai yang sesuai \textit{personal quality} \\
		\hline
		
	\end{tabular}
\end{table}

\item Skenario Cari Kandidat

Nomor \kern 3.6pc : SP-33 \\
Nama Use Case : Mencari kandidat baru \\
Aktor \kern 4.1 pc : \textit{User} \\
Tipe \kern 4.6pc : \textit{Primary} \\
Tujuan \kern 3.6pc : \textit{User} dapat mencari kandidat baru \\
Deskripsi \kern 2.5pc : 

\begin{itemize}
	\item \textit{User} menuju ke halaman cari kandidat
	\item Sistem akan menampilkan halaman cari kandidat
	\item \textit{User} mengisi persyaratan umum
	\item Sitem menampilkan daftar data kandidat
	\item \textit{User} melakukan \textit{filtering}
	\item Sitem menampilkan daftar data kandidat sesuai \textit{filtering}
	
\end{itemize}

\begin{table}
	\caption{Skenario Cari Kandidat}
	\centering
	\begin{tabular}{ | l | p{67.5mm}|}
		\hline 
		\textbf{Aktor} & \textbf{Sistem} \\
		\hline
		
		1.	Menuju ke halaman cari kandidat &  \\
		
		\hline
		
		&  2.	Menampilkan halaman cari kandidat \\
		
		\hline
		
		3. Mengisi persyaratan umum & \\
		
		\hline
		
		& 4. Menampilkan daftar data kandidat \\
		\hline
		
		5. Melakukan \textit{filtering}
		
		& 6. Menampilkan daftar data kandidat sesuai \textit{filtering} \\
		\hline
		
	\end{tabular}
\end{table}

\item Skenario Memilih Kandidat

Nomor \kern 3.6pc : SP-34 \\
Nama Use Case : Memilih kandidat \\
Aktor \kern 4.1 pc : \textit{User} \\
Tipe \kern 4.6pc : \textit{Primary} \\
Tujuan \kern 3.6pc : \textit{User} dapat melakukan pemilihan kandidat \\
Deskripsi \kern 2.5pc : 

\begin{itemize}
	\item \textit{User} menuju ke halaman cari kandidat
	\item Sistem akan menampilkan halaman cari kandidat
	\item \textit{User} mengisi persyaratan umum
	\item Sitem menampilkan daftar data kandidat
	\item \textit{User} melakukan \textit{filtering}
	\item Sitem menampilkan daftar data kandidat sesuai \textit{filtering}
	\item \textit{User} memilih kandidat
	\item Sistem menampilkan \textit{pop-up} tanda pemilihan kandidat berhasil
	
\end{itemize}

\begin{table}
	\caption{Skenario Memilih Kandidat}
	\centering
	\begin{tabular}{ | l | p{67.5mm}|}
		\hline 
		\textbf{Aktor} & \textbf{Sistem} \\
		\hline
		
		1.	Menuju ke halaman cari kandidat &  \\
		
		\hline
		
		&  2.	Menampilkan halaman cari kandidat \\
		
		\hline
		
		3. Mengisi persyaratan umum & \\
		
		\hline
		
		& 4. Menampilkan daftar data kandidat \\
		\hline
		
		5. Melakukan \textit{filtering} & \\
		\hline
		
		& 6. Menampilkan daftar data kandidat sesuai \textit{filtering} \\
		\hline
		
		
		7. Memilih kandidat & \\
		\hline
		
		&8. Menampilkan \textit{pop-up} tanda berhasil memilih kandidat \\
		\hline
		
	\end{tabular}
\end{table}

\item Skenario \textit{View} Kandidat Sementara

Nomor \kern 3.6pc : SP-35 \\
Nama Use Case : Melakukan \textit{view} kandidat sementara \\
Aktor \kern 4.1 pc : \textit{User} \\
Tipe \kern 4.6pc : \textit{Primary} \\
Tujuan \kern 3.6pc : \textit{User} dapat melihat data kandidat sementara \\
Deskripsi \kern 2.5pc : 

\begin{itemize}
	\item \textit{User} menuju ke \textit{icon} orang
	\item Sistem akan menampilkan \textit{dropdown} kandidat sementara
	\item \textit{User} memilih \textit{view} kandidat sementara
	\item Sitem menampilkan data kandidat sementara
	\item \textit{User} melihat data kandidat sementara

	
\end{itemize}

\begin{table}
	\caption{Skenario \textit{View} Kandidat Sementara}
	\centering
	\begin{tabular}{ | l | p{66mm}|}
		\hline 
		\textbf{Aktor} & \textbf{Sistem} \\
		\hline
		
		1.	Menuju ke \textit{icon} orang&  \\
		
		\hline
		
		&  2.	Menampilkan halaman  \textit{dropdown} kandidat sementara \\
		
		\hline
		
		3. Memilih \textit{view} kandidat sementara & \\
		
		\hline
		
		& 4. Menampilkan data kandidat sementara \\
		\hline
		
		5. Melihat data kandidat sementara & \\
		\hline
		
		
	\end{tabular}
\end{table}

\item Skenario Memproses Kandidat Sementara

Nomor \kern 3.6pc : SP-36 \\
Nama Use Case : Memproses kandidat sementara \\
Aktor \kern 4.1 pc : \textit{User} \\
Tipe \kern 4.6pc : \textit{Primary} \\
Tujuan \kern 3.6pc : \textit{User} dapat memproses kandidat sementara \\
Deskripsi \kern 2.5pc : 

\begin{itemize}
	\item User menuju ke \textit{icon} orang
	\item Sistem akan menampilkan \textit{dropdown} kandidat sementara
	\item User memilih \textit{view} kandidat sementara
	\item Sitem menampilkan data kandidat sementara
	\item User memilih proses kandidat sementara
	\item Sistem menampilkan \textit{pop-up} berhasil memproses kandidat sementara
	
\end{itemize}

\begin{table}
	\caption{Skenario Proses Kandidat Sementara}
	\centering
	\begin{tabular}{ | l | p{64mm}|}
		\hline 
		\textbf{Aktor} & \textbf{Sistem} \\
		\hline
		
			1.	Menuju ke \textit{icon} orang&  \\
			
			\hline
			
			&  2.	Menampilkan halaman  \textit{dropdown} kandidat sementara \\
			
			\hline
			
			3. Memilih \textit{view} kandidat sementara & \\
			
			\hline
			
			& 4. Menampilkan data kandidat sementara \\
			\hline
			
			
			5. Memilih proses kandidat sementara &  \\
			\hline
			
			& 6. Menampilkan \textit{pop-up }berhasil memproses kandidat sementara \\
			\hline
			
	\end{tabular}
\end{table}

\end{enumerate}

\subsection{Class Diagram}

\begin{figure}
	\centering
	\includegraphics[width=1\textwidth]
	{pics/diagram/classdiagram.png}
	\caption{Class Diagram}
	\label{fig:32}
\end{figure}

Terdapat banyak relasi antar tabel pada gambar diatas, terutama pada tabel pegawai yang hampir mempunyai relasi di setiap tabel. Data pegawai harus sedemikian lengkap guna menunjang proses \textit{filtering}. Data pegawai berupa data jabatan, riwayat pendidikan, \textit{skill},\textit{ personal quality} dan data pribadi lainnya. 

\subsection{Entity Relationship Diagram}

\begin{figure}
	\centering
	\includegraphics[width=0.9\textwidth]
	{pics/diagram/erd.png}
	\caption{ERD}
	\label{fig:32}
\end{figure}

%-----------------------------------------------------------------------------%
\section{Perancangan Aplikasi}
%-----------------------------------------------------------------------------%
Dalam perancangan aplikasi \textbf{SiPJabS} diperlukan perancangan antar muka dan perancangan design level tinggi. Perancangan antar muka akan menjelaskan gambaran awal  \textbf{developer} sebelum masuk pada bagian \textit{front-end}.  Sedangkan perancangan \textit{design} level tinggi berguna untuk mengingatkan \textit{developer} tentang sistem kerja pada aplikasi yang akan dibuat.


\subsection{Perancangan Antar Muka}
Pada tahap kebutuhan antar muka terdapat gambaran mengenai aplikasi \textbf{"SiPJabS: Sistem Pengawakan Jabatan Struktural"}, berikut merupakan mockup dari aplikasi \textbf{SiPJabS} yang sudah dibuat.

\subsubsection{Perancangan Antar Muka Admin}

\begin{table}
	\caption{Tabel Perancangan Antar Muka Admin}
	\centering
	\begin{tabular}{ | c | c | p{35mm} |}
		\hline 
		\textbf{No} & \textbf{Gambar} &  \textbf{Keterangan} \\ 
		\hline
		
		1. & \raisebox{-\totalheight}{\includegraphics[width=0.6\textwidth, height=60mm]{pics/admin/login.png}} 
		& Halaman \textit{login} merupakan tampilan awal apabila admin membuka aplikasi \textbf{SiPJabS} , admin dapat menginputkan \textit{username} dan \textit{password} untuk melakukan \textit{login}. \\
	
		\hline
		
		2. & \raisebox{-\totalheight}{\includegraphics[width=0.6\textwidth, height=60mm]{pics/admin/dashboard.png}} 
		& Didalam \textit{dashboard} admin terdapat jumlah \textit{users}, jumlah pegawai, kandidat yang sudah dipilih, unit kerja, jabatan, unit bagian, pendidikan, \textit{skill} dan \textit{personal quality} yang dimiliki para pegawai  Universitas Telkom. \\
		\hline

	\end{tabular}
\end{table}


\begin{table}
	\caption{Tabel Perancangan Antar Muka Admin (1)}
	\centering
	\begin{tabular}{ | c | c | p{35mm} |}
		\hline 
		\textbf{No} & \textbf{Gambar} &  \textbf{Keterangan} \\ 
		\hline
		
		3. & \raisebox{-\totalheight}{\includegraphics[width=0.6\textwidth, height=60mm]{pics/admin/profile.png}} 
		& Halaman \textit{profile} admin akan menampilkan data \textit{profile} dari admin tersebut. Kemudian admin juga dapat mengedit \textit{profile} dan me\textit{reset} \textit{password}.  \\
		
		\hline
		
		4. & \raisebox{-\totalheight}{\includegraphics[width=0.6\textwidth, height=60mm]{pics/admin/editprofile.png}} 
		& Admin dapat mengubah \textit{username}, menginputkan \textit{email}, dan menambahkan foto \textit{profile}.  \\
		
		\hline
		
		5. & \raisebox{-\totalheight}{\includegraphics[width=0.6\textwidth, height=60mm]{pics/admin/resetpassword.png}} 
		& Admin harus menginputkan \textit{password} yang lama serta yang baru, setelah itu admin dapat mengubah \textit{password}. \\
		
		\hline
		
	\end{tabular}
\end{table}

\begin{table}
	\caption{Tabel Perancangan Antar Muka Admin (2)}
	\centering
	\begin{tabular}{ | c | c | p{35mm} |}
		\hline 
		\textbf{No} & \textbf{Gambar} &  \textbf{Keterangan} \\ 
		\hline
		
		
		
		6. & \raisebox{-\totalheight}{\includegraphics[width=0.6\textwidth, height=60mm]{pics/admin/help.png}} 
		& Halaman \textit{help} berisi infomasi tentang aplikasi. \\
		
		\hline
		
		7. & \raisebox{-\totalheight}{\includegraphics[width=0.6\textwidth, height=60mm]{pics/admin/datausers.png}} 
		& Halaman data \textit{user} akan menampilkan nama-nama yang dapat mengakses aplikasi \textbf{SiPJabS} sebagai admin dan \textit{user}. \\
		
		\hline
		
		8. & \raisebox{-\totalheight}{\includegraphics[width=0.6\textwidth, height=60mm]{pics/admin/editdatausers.png}} 
		& Pada halaman ini admin dapat mengedit \textit{username}, \textit{email}, dan \textit{role} sebagai admin atau \textit{user}. \\
		
		\hline
		
	\end{tabular}
\end{table}

\begin{table}
	\caption{Tabel Perancangan Antar Muka Admin (3)}
	\centering
	\begin{tabular}{ | c | c | p{35mm} |}
		\hline 
		\textbf{No} & \textbf{Gambar} &  \textbf{Keterangan} \\ 
		\hline
		
		9. & \raisebox{-\totalheight}{\includegraphics[width=0.6\textwidth, height=60mm]{pics/admin/tambahusers.png}} 
		& Admin dapat menambahkan \textit{user} dengan mengisi \textit{form} tambah \textit{user} dan menyimpannya. \\
		
		\hline
		
		10. & \raisebox{-\totalheight}{\includegraphics[width=0.6\textwidth, height=60mm]{pics/admin/datapegawai.png}} 
		& Admin dapat melihat daftar data pegawai yang ada di Universitas Telkom secara detail. \\
		
		\hline
		
		11. & \raisebox{-\totalheight}{\includegraphics[width=0.6\textwidth, height=60mm]{pics/admin/editpegawai.png}} 
		& Admin dapat mengedit data pegawai secara lengkap. \\
		
		\hline
		
	\end{tabular}
\end{table}

\begin{table}
	\caption{Tabel Perancangan Antar Muka Admin (4)}
	\centering
	\begin{tabular}{ | c | c | p{35mm} |}
		\hline 
		\textbf{No} & \textbf{Gambar} &  \textbf{Keterangan} \\ 
		\hline
		
		
		
		12. & \raisebox{-\totalheight}{\includegraphics[width=0.6\textwidth, height=60mm]{pics/admin/viewpegawai.png}} 
		&Admin dapat melihat data pribadi pegawai secara lengkap dan data detail informasi lainnya. \\
		
		\hline
		
		13. & \raisebox{-\totalheight}{\includegraphics[width=0.6\textwidth, height=60mm]{pics/admin/datatallent.png}} 
		& Halaman ini akan menampilkan data kandidat yang sudah di pilih oleh \textit{user} sesuai dengan \textit{job description} untuk menggantikan atau mengisi posisi yang kosong. \\
		
		\hline
		
		14. & \raisebox{-\totalheight}{\includegraphics[width=0.6\textwidth, height=60mm]{pics/admin/viewdetailtallent.png}} 
		& Admin dapat melihat data detail kandidat yang sudah dipilih. \\
		
		\hline
		
	\end{tabular}
\end{table}

\begin{table}
	\caption{Tabel Perancangan Antar Muka Admin (5)}
	\centering
	\begin{tabular}{ | c | c | p{35mm} |}
		\hline 
		\textbf{No} & \textbf{Gambar} &  \textbf{Keterangan} \\ 
		\hline
		
		
		
		15. & \raisebox{-\totalheight}{\includegraphics[width=0.6\textwidth, height=60mm]{pics/admin/dataunitkerja.png}} 
		&Halaman ini akan menunjukkan semua unit kerja dimulai dari rektorat hingga fakultas yang ada di Universitas Telkom. \\
		
		\hline
		
		16. & \raisebox{-\totalheight}{\includegraphics[width=0.6\textwidth, height=60mm]{pics/admin/editunitkerja.png}} 
		& Admin dapat mengedit \textit{form} unit kerja apabila terdapat kebijakan baru. \\
		
		\hline
		
		17. & \raisebox{-\totalheight}{\includegraphics[width=0.6\textwidth, height=60mm]{pics/admin/tambahunitkerja.png}} 
		& Admin dapat menambahkan data unit kerja dengan mengisi \textit{form} tersebut, namun harus sesuai dengan kebijakan yang telah ditetapkan. \\
		
		\hline
		
	\end{tabular}
\end{table}

\begin{table}
	\caption{Tabel Perancangan Antar Muka Admin (6)}
	\centering
	\begin{tabular}{ | c | c | p{35mm} |}
		\hline 
		\textbf{No} & \textbf{Gambar} &  \textbf{Keterangan} \\ 
		\hline
		
	
		
		18. & \raisebox{-\totalheight}{\includegraphics[width=0.6\textwidth, height=60mm]{pics/admin/datajabatan.png}} 
		&Halaman ini akan menampilkan data jabatan yang berada di Universitas Telkom \\
		
		\hline
		
		19. & \raisebox{-\totalheight}{\includegraphics[width=0.6\textwidth, height=60mm]{pics/admin/editjabatan.png}} 
		& Admin dapat mengedit data jabatan sesuai dengan nama jabatan yang sudah ditetapkan. \\
		
		\hline
		
		20. & \raisebox{-\totalheight}{\includegraphics[width=0.6\textwidth, height=60mm]{pics/admin/tambahjabatan.png}} 
		& Admin harus melengkapi \textit{form} tersebut untuk dapat menambahkan data jabatan yang baru. \\
		
		\hline
		
	\end{tabular}
\end{table}

\begin{table}
	\caption{Tabel Perancangan Antar Muka Admin (7)}
	\centering
	\begin{tabular}{ | c | c | p{35mm} |}
		\hline 
		\textbf{No} & \textbf{Gambar} &  \textbf{Keterangan} \\ 
		\hline
		
		
		
		21. & \raisebox{-\totalheight}{\includegraphics[width=0.6\textwidth, height=60mm]{pics/admin/dataunitbagian.png}} 
		&Halaman ini akan menampilkan satuan kerja yang terdapat di Universitas Telkom.  \\
		
		\hline
		
		22. & \raisebox{-\totalheight}{\includegraphics[width=0.6\textwidth, height=60mm]{pics/admin/editunitbagian.png}} 
		& Admin dapat mengedit data unit bagian apabila ada perubahan yang sudah ditetapkan.\\
		
		\hline
		
		23. & \raisebox{-\totalheight}{\includegraphics[width=0.6\textwidth, height=60mm]{pics/admin/tambahunitbagian.png}} 
		& Admin harus menginputkan  nama unit bagian  tersebut untuk dapat menambahkan data data unit bagian yang baru. \\
		
		\hline
		
	\end{tabular}
\end{table}

\begin{table}
	\caption{Tabel Perancangan Antar Muka Admin (8)}
	\centering
	\begin{tabular}{ | c | c | p{35mm} |}
		\hline 
		\textbf{No} & \textbf{Gambar} &  \textbf{Keterangan} \\ 
		\hline
		
		
		
		24. & \raisebox{-\totalheight}{\includegraphics[width=0.6\textwidth, height=60mm]{pics/admin/datajabstruk.png}} 
		&Halaman ini menampilkan jabatan yang secara tegas ada di Universitas Telkom.  \\
		
		\hline
		
		25. & \raisebox{-\totalheight}{\includegraphics[width=0.6\textwidth, height=60mm]{pics/admin/editjabstruk.png}} 
		& Admin dapat mengedit data dan harus mengisi \textit{form} sesuai dengan ketetapan.\\
		
		\hline
		
		26. & \raisebox{-\totalheight}{\includegraphics[width=0.6\textwidth, height=60mm]{pics/admin/tambahjabstruk.png}} 
		& Admin harus melengkapi \textit{form} untuk dapat menambahkan data jabatan struktural baru yang sudah ditetapkan. \\
		
		\hline
		
	\end{tabular}
\end{table}

\begin{table}
	\caption{Tabel Perancangan Antar Muka Admin (9)}
	\centering
	\begin{tabular}{ | c | c | p{35mm} |}
		\hline 
		\textbf{No} & \textbf{Gambar} &  \textbf{Keterangan} \\ 
		\hline
		
		
		
		27. & \raisebox{-\totalheight}{\includegraphics[width=0.6\textwidth, height=60mm]{pics/admin/datapendidikan.png}} 
		&Halaman ini akan menampilkan data pendidikan yang dimiliki pegawai Universitas Telkom.  \\
		
		\hline
		
		28. & \raisebox{-\totalheight}{\includegraphics[width=0.6\textwidth, height=60mm]{pics/admin/editpendidikan.png}} 
		& Apabila ingin mengedit maka admin harus menginputkan jenjang pendidikan serta jurusan. \\
		
		\hline
		
		29. & \raisebox{-\totalheight}{\includegraphics[width=0.6\textwidth, height=60mm]{pics/admin/tambahpendidikan.png}} 
		& Admin dapat menambahkan data pendidikan apabila belum ada data pendidikan yang dimiliki pegawai belum terinput. \\
		
		\hline
		
	\end{tabular}
\end{table}

\begin{table}
	\caption{Tabel Perancangan Antar Muka Admin (10)}
	\centering
	\begin{tabular}{ | c | c | p{35mm} |}
		\hline 
		\textbf{No} & \textbf{Gambar} &  \textbf{Keterangan} \\ 
		\hline
		
		
		
		30. & \raisebox{-\totalheight}{\includegraphics[width=0.6\textwidth, height=60mm]{pics/admin/dataskill.png}} 
		&Halaman ini akan menampilkan \textit{skill} yang dimiliki pegawai Universitas Telkom untuk menunjang pekerjaan.  \\
		
		\hline
		
		31. & \raisebox{-\totalheight}{\includegraphics[width=0.6\textwidth, height=60mm]{pics/admin/editskill.png}} 
		& Admin dapat mengedit data \textit{skill}. \\
		
		\hline
		
		32. & \raisebox{-\totalheight}{\includegraphics[width=0.6\textwidth, height=60mm]{pics/admin/tambahskill.png}} 
		& Admin dapat menambahkan data \textit{skill} baru apabila terdapat data \textit{skill} yang belum diinputkan.  \\
		
		\hline
		
	\end{tabular}
\end{table}

\begin{table}
	\caption{Tabel Perancangan Antar Muka Admin (11)}
	\centering
	\begin{tabular}{ | c | c | p{35mm} |}
		\hline 
		\textbf{No} & \textbf{Gambar} &  \textbf{Keterangan} \\ 
		\hline
		
		
		
		33. & \raisebox{-\totalheight}{\includegraphics[width=0.6\textwidth, height=60mm]{pics/admin/datapersonalquality.png}} 
		&Halaman ini akan menampilkan \textit{personal quality} yang dimiliki pegawai Universitas Telkom untuk menunjang pekerjaan.  \\
		
		\hline
		
		34. & \raisebox{-\totalheight}{\includegraphics[width=0.6\textwidth, height=60mm]{pics/admin/editpersonalquality.png}} 
		& Admin dapat mengedit data \textit{personal quality}. \\
		
		\hline
		
		36. & \raisebox{-\totalheight}{\includegraphics[width=0.6\textwidth, height=60mm]{pics/admin/tambahpersonalquality.png}} 
		& Admin dapat menambahkan data \textit{personal quality} baru apabila terdapat data \textit{personal quality} yang belum diinputkan.  \\
		
		\hline
		
	\end{tabular}
\end{table}

\subsubsection{Perancangan Antar Muka User}

\begin{table}
	\caption{Tabel Perancangan Antar Muka User}
	\centering
	\begin{tabular}{ | c | c | p{35mm} |}
		\hline 
		\textbf{No} & \textbf{Gambar} &  \textbf{Keterangan} \\ 
		\hline
		
		1. & \raisebox{-\totalheight}{\includegraphics[width=0.6\textwidth, height=60mm]{pics/user/login.png}} 
		& Halaman \textit{login} merupakan tampilan awal apabila \textit{user} membuka aplikasi \textit{SiPJabS} ,\textit{user} dapat menginputkan \textit{username} dan \textit{password} untuk melakukan \textit{login}.  \\
		
		\hline
		
		2. & \raisebox{-\totalheight}{\includegraphics[width=0.6\textwidth, height=60mm]{pics/user/dashboard.png}} 
		& Halaman \textit{dashboard} akan menampilkan jumlah \textit{user} yang dapat mengakses aplikasi \textbf{SiPJabS}, jumlah pegawai yang ada di Universitas Telkom serta jumlah kandidat yang sudah dipilih.  \\
		
		\hline
		
		3. & \raisebox{-\totalheight}{\includegraphics[width=0.6\textwidth, height=60mm]{pics/user/profile.png}} 
		& Halaman \textit{profile user} akan menampilkan data nama lengkap, \textit{username}, \textit{email}, status pegawai, unit kerja, jabatan, serta NIP dari \textit{user} tersebut. Kemudian \textit{user} juga dapat mengedit \textit{profile} dan me\textit{reset} \textit{password}.\\
		
		\hline
		
		
	\end{tabular}
\end{table}

\begin{table}
	\caption{Tabel Perancangan Antar Muka User (1)}
	\centering
	\begin{tabular}{ | c | c | p{35mm} |}
		\hline 
		\textbf{No} & \textbf{Gambar} &  \textbf{Keterangan} \\ 
		\hline
		
		4. & \raisebox{-\totalheight}{\includegraphics[width=0.6\textwidth, height=60mm]{pics/user/editprofile.png}} 
		& \textit{User} dapat mengubah \textit{username}, menginputkan \textit{email}, dan menambahkan foto \textit{profile}  \\
		
		\hline
		
		5. & \raisebox{-\totalheight}{\includegraphics[width=0.6\textwidth, height=60mm]{pics/user/resetpassword.png}} 
		& User harus menginputkan \textit{password} yang lama serta yang baru, setelah itu user dapat me\textit{reset password}.  \\
		
		\hline
		
		6. & \raisebox{-\totalheight}{\includegraphics[width=0.6\textwidth, height=60mm]{pics/user/help.png}} 
		& Pada halaman \textit{help} akan mengenai informasi dari sistem aplikasi ini\\
		
		\hline
		
	\end{tabular}
\end{table}

\begin{table}
	\caption{Tabel Perancangan Antar Muka User (2)}
	\centering
	\begin{tabular}{ | c | c | p{35mm} |}
		\hline 
		\textbf{No} & \textbf{Gambar} &  \textbf{Keterangan} \\ 
		\hline
		
		7. & \raisebox{-\totalheight}{\includegraphics[width=0.6\textwidth, height=60mm]{pics/user/caritallent.png}} 
		& \textit{User} dapat memilih jabatan dan masa kerja yang diinginkan untuk mengantikan atau mengisi posisi yang kosong.  \\
		
		\hline
		
		8. & \raisebox{-\totalheight}{\includegraphics[width=0.6\textwidth, height=60mm]{pics/user/hasiltallent.png}} 
		& Halaman ini akan menampilkan kandidat yang sudah dipilih dengan syarat tententu.   \\
		
		\hline
		
		9. & \raisebox{-\totalheight}{\includegraphics[width=0.6\textwidth, height=60mm]{pics/user/viewdetailtallent.png}} 
		& Halaman ini akan menampilkan data pribadi dari kandidat tersebut.\\
		
		\hline
		
	\end{tabular}
\end{table}

\begin{table}
	\caption{Tabel Perancangan Antar Muka User (3)}
	\centering
	\begin{tabular}{ | c | c | p{35mm} |}
		\hline 
		\textbf{No} & \textbf{Gambar} &  \textbf{Keterangan} \\ 
		\hline
		
		10. & \raisebox{-\totalheight}{\includegraphics[width=0.6\textwidth, height=60mm]{pics/user/datatallent.png}} 
		& Halaman ini akan menampilkan data kandiat yang sudah diproses dari kandidat sementara tersebut.  \\
		
	
		\hline
		
		11. & \raisebox{-\totalheight}{\includegraphics[width=0.6\textwidth, height=60mm]{pics/user/cartuser.png}} 
		& Halaman ini akan menampilkan data kandidat sementara yang sudah dipilih dan masih ada kemungkinan bisa diubah sebelum ditetapkan menjadi kandidat.  \\
		
		\hline
		
		12. & \raisebox{-\totalheight}{\includegraphics[width=0.6\textwidth, height=60mm]{pics/user/datakandidatsementara.png}} 
		& Halaman ini akan menampilkan data kandidat sementara yang sudah dipilih dan akan diproses ditetapkan untuk menjadi kandidat.  \\
		
		
		\hline
		
	\end{tabular}
\end{table}

\subsection{Perancangan Level Tinggi}

\begin{figure}
	\centering
	\includegraphics[width=1\textwidth]
	{pics/highleveldesign.png}
	\caption{High Level Design}
	\label{fig:34}
\end{figure}

Alur perancanaan level tinggi pada aplikasi \textbf{"SiPJabS : Sistem Pengawakan Jabatan Struktural"} dimulai pengguna melakukan perintah dah akan di salurkan melalui HTTP \textit{Request} ke \textit{server}. Pengambilan data akan di \textit{filter} berdasarkan dengan \textit{query} yang dibuat berdasarkan data yang diperoleh dari \textit{database} yang ada. Kemudian \textit{database} akan memberikan umpan balik berupa HTTP Response berdasarkan \textit{request} data yang akan ditampilkan kepada pengguna. 

