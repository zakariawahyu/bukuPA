%---------------------------------------------------------------
\chapter{\babLima}
%---------------------------------------------------------------

%---------------------------------------------------------------
\section{Kesimpulan}
%---------------------------------------------------------------
Berdasarkan rancangan dan dari hasil analisa pengujian aplikasi yang telah dilakukan, adapun kesimpulan aplikasi Sistem Pengawakan Jabatan Struktural adalah sebagai berikut :

\begin{enumerate}
	\item Aplikasi SiPJabS dibangun dengan menggunakan back-end framework laravel, fornt-end menggunakan bootsrap dan javascript serta dabatase menggunakan mysql. Dan di hosting menggunakan server indonesia yang cukup stabil.
	
	\item Aplikasi SiPJabS dapat membantu pegawai direktorat sumber daya manusia universitas telkom bagian pengembangan karir dalam mencari kandidat baru secara cepat dan efisien.
	
	\item Hasil dari responden adalah setuju dengan fitur aplikasi yang sudah
	dibuat sesuai dengan tujuan aplikasi.
	
\end{enumerate}


%---------------------------------------------------------------
\section{Saran}
%---------------------------------------------------------------
Berdasarkan hasil dari pengujian usability testing, terdapat beberapa saran untuk pengembangan aplikasi SiPJabS agar lebih baik lagi. Adapun saran adalah sebagai berikut:

\begin{enumerate}
	\item Karena keterbatasan penyampaian data dari pengembangan karir, aplikasi SiPJabS secara kelengkapan data belum lengkap dan belum dapat digunakan secara maksimal. Untuk pengembangan selanjutnya, pengumpulan data dapat dibuka saat bekerja di kantor dan diawasi oleh Kepala Urusan Pengembangan Karir.
	
	\item Penambahan fitur tambah data secara banyak menggunakan excel, sehingga tidak perlu menginputkan satu-satu.
\end{enumerate}