
\chapter*{Abstrak}
\vspace*{0.7cm}

Banyak model seleksi yang dilakukan guna untuk menilai seseorang terutama ketika perusahaan mencari posisi jabatan diantaranya dengan melakukan assessment center dan mengisi formulir penilaian untuk setiap kandidat yang akan dicalonkan sebagai pemimpin dan staf. Banyak prosedur serta ketentuan yang harus dimiliki oleh calon pemimpin dan staf, baik itu manajer atau kepala bagian dan staf. Setiap orang yang terpilih berarti telah memenuhi ketentuan yang sudah ditetapkan perusahaan. Ketentuan dibuat berdasarkan kompetensi setiap bagian yang disusun dalam kamus kompetensi perusahaan. Melalui kamus kompetensi tersebut juga dapat dijadikan sebagai pedoman untuk bagian Sumber Daya Manusia dalam mencari pegawai yang berpotensi tinggi demi keberlangsungan perusahaan. Terdapat permasalahan belum adanya proses mekanisme penentuan kandidat yang tepat, apabila terdapat posisi yang digantikan atau kosong. Maka tidak adanya data pegawai yang akan dijadikan kandidat untuk mengisi posisi yang digantikan atau kosong tersebut.
\\

Untuk mengatasi permasalahan diatas, maka di rancang aplikasi \textbf{”SiPJabS : Sistem Pengawakan Jabatan Struktural di Universitas Telkom”} yang bertujuan untuk memperbaiki proses manajemen karir. Agar organisasi atau perusahaan memperoleh kandidat yang berkualitas. Dengan memberikan penerapan lebih kompetetif dan adil. Kemudian juga dapat menganalisis risiko, misalnya identifikasi pegawai yang berpotensi keluar. Maka dapat meningkatkan program pembelajaran dan pengembangan untuk kinerja dan mengembangkan kompetensi yang lebih baik di masa depan. 

\vspace*{0.2cm}

\noindent \textbf{Kata Kunci:} \textit{Pegawai}, \textit{Perekrutan}, \textit{Perusahaan}\\ 

\newpage