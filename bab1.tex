%-----------------------------------------------------------------------------%
\chapter{\babSatu}
%-----------------------------------------------------------------------------%
%-----------------------------------------------------------------------------%
\section{Latar Belakang}
%-----------------------------------------------------------------------------%
Keberlangsungan perguruan tinggi Universitas Telkom tak akan lepas dari peran orang-orang yang bekerja didalamnya. Dengan struktur organisasi yang kompleks, menjadikan Universitas Telkom terdepan dibidangnya. Setiap pemegang jabatan memiliki peran yang penting dalam menunjang visi dan misi Universitas Telkom untuk mencapai tujuannya. Oleh karena itu, setiap orang yang terpilih untuk memegang jabatan penting di perguruan tinggi ini pasti memiliki kompetensi yang sesuai dengan apa yang diharapkan. Orang-orang tersebut dapat terpilih melalui tahap seleksi yang panjang, agar perguruan tinggi Universitas Telkom mendapatkan orang-orang terbaik untuk menjalankan tugasnya.

Banyak model seleksi yang dilakukan untuk menilai seseorang terutama ketika perusahaan mencari seorang pemimpin dan staf, diantaranya dengan melakukan assessment center dan mengisi formulir penilaian untuk setiap kandidat yang akan dicalonkan sebagai pemimpin dan staf. Banyak prosedur serta ketentuan yang harus dimiliki oleh calon pemimpin dan staf, baik itu manajer, kepala bagian, kepala urusan, sekretaris atau staf. Setiap orang yang terpilih berarti telah memenuhi ketentuan yang sudah ditetapkan perusahaan. Ketentuan dibuat berdasarkan kompetensi setiap bagian yang disusun dalam kamus kompetensi perusahaan. Melalui kamus kompetensi tersebut juga dapat dijadikan sebagai pedoman untuk bagian Sumber Daya Manusia dalam mencari pegawai yang berpotensi tinggi demi keberlangsungan perusahaan.

Masalah yang paling banyak dijumpai pada suatu perusahaan yaitu berkaitan dengan pencarian kandidat yang sesuai dengan job description. Banyak kandidat memiliki skill yang sama dengan kandidat lainnya, namun perusahaan mencari kandidat sesuai dengan requitment yang telah ditetapkan pada perusahaan. Dengan begitu, proses filtering akan membutuhkan waktu yang lama, jika job description tidak cepat ditemukan. Untuk mengatasi permasalahan tersebut, proses filtering akan dipindahkan dengan aplikasi “\textbf{SiPJabS: Sistem Pengawakan Jabatan Struktural}” , yang diharapkan dapat membantu penemuan kandidat yang sesuai dengan requitment yang sudah ditetapkan oleh perusahaan.
\\

Terdapat permasalahan belum adanya proses mekanisme penentuan kandidat yang tepat, apabila terdapat posisi yang kosong. Maka tidak adanya data pegawai yang akan dijadikan kandidat untuk mengisi posisi yang kosong tersebut, sehingga perlu dirancang sistem yang mampu mengidentifikasi sesuai kebutuhan posisi yang diinginkan. Yang kemudian hasilnya akan disesuaikan dengan job description yang dibutuhkan oleh perusahaan. Dengan cara ini, manajer di perusahaan dapat menentukan profil pegawai yang tepat sesuai requitment yang sudah ditetapkan oleh perusahaan.

Untuk mengatasi permasalahan di atas, kami merancang Proyek Akhir ini dengan membuat aplikasi staffing karir yang bertujuan untuk untuk mengelola proses pencarian kandidat yang akan dicalonkan sebagai pemimpin dan staf, agar perusahaan memperoleh kandidat yang memenuhi ketentuan yang sudah ditetapkan perusahaan dengan menerapkan penilaian yang lebih lengkap dan adil. Diharapkan dengan adanya sistem ini, manajer mampu menganalisa kebutuhan program pengembangan kompetensi Sumber Daya Manusia yang lebih baik di masa depan.


%-----------------------------------------------------------------------------%
\section{Perumusan Masalah}
%-----------------------------------------------------------------------------%
Berdasarkan latar belakang di atas, maka rumusan masalah yang akan dibahas adalah sebagai berikut:
\begin{enumerate}
\item Bagaimana proses pengisian posisi jabatan yang tepat ?
\item Bagaimana cara pencarian data kinerja pegawai yang sesuai requitment ?
\item Bagaimana proses \textit{filtering} berjalan efektif
\end{enumerate}

%-----------------------------------------------------------------------------%
\section{Batasan Permasalahan}
%-----------------------------------------------------------------------------%
Batasan masalah yang terdapat dapat dari perumusan masalah adalah sebagai berikut :

\begin{enumerate}
	\item Aplikasi ini hanya dapat digunakan oleh pengguna yang bertugas   mengelola sistem filtering yaitu pegawai Direktorat Sumber Daya Manusia Telkom University.
	\item Penilaian kinerja ini dibangun dengan menggunakan sistem berbasis web
	\item Setiap pemilihan jabatan memiliki parameter berbeda, yang mengacu pada metode penilaian.
\end{enumerate}

%-----------------------------------------------------------------------------%
\section{Tujuan}
%-----------------------------------------------------------------------------%
Tujuan yang ingin dicapai dari perancangan Proyek Akhir ini adalah :

\begin{enumerate}
	\item Membangun aplikasi pengawakan jabatan struktural yang bertujuan untuk mengelola proses pencarian kandidat yang akan dicalonkan sebagai pemimpin dan staf, agar perusahaan memperoleh kandidat yang memenuhi ketentuan yang sudah ditetapkan perusahaan dengan menerapkan penilaian yang lebih lengkap dan adil. Diharapkan dengan adanya sistem ini, manajer mampu menganalisa kebutuhan program pengembangan kompetensi Sumber Daya Manusia yang lebih baik di masa depan.
	\item Adanya sistem pengawakan jabatan struktural, user dapat menggunakan aplikasi setiap saat untuk mencari kandidat dan mengisi posisi yang kosong karena menggantikan pekerjaan lama yang telah berhenti dikarenakan pensiun, meninggal, mengundurkan diri atau diberhentikan karena suatu kebijakan tertentu.
\end{enumerate}

%-----------------------------------------------------------------------------%
\section{Metode Penyelesaian Masalah}
%-----------------------------------------------------------------------------%
Metodologi untuk menyelesaikan masalah diatas adalah sebagai berikut :

\begin{enumerate}
	\item Tahap studi literatur \\
	Tahap pertama ini dilakukan dengan cara mencari, menganalisa dan 		mempelajari informasi yang berhubungan dengan Proyek Akhir. Topik yang berhubungan antara lain: 
	\begin{enumerate}
	\item Data pegawai secara lengkap.
	\item Data posisi yang mencari kandidat calon pemimpin. 
	\end{enumerate}
	Serta teori lain yang berhubungan dengan pengembangan aplikasi. Referensi dapat dicari melalui buku, jurnal, paper, dan media lainnya baik daring maupun luring
	\item	Tahap pencarian dan pengumpulan data \\
	Pencarian dan pengumpulan data yang diperlukan dalam pengembangan aplikasi ini seperti data yang terdapat pada I-GRACIAS dan data yang terdapat pada Sumber Daya Manusia.
	\\
	\item	Tahap perancangan sistem \\
	Perancangan sistem aplikasi ini dimulai dengan perancangan mockup atau desain UI aplikasi dan UX aplikasi serta merancang database dan kerangka program yang akan digunakan.
	\item Tahap implementasi \\
	Tahap ini dilakukan realisasi dari perancangan sistem yang telah dibuat, seperti membuat prototype dan UI dari aplikasi, pembuatan database dan aplikasi yang sudah direncanakan pada tahap perancangan sistem.
	\item Tahap pengujian dan analisis \\
	Pengujian dan analisis ini dilakukan apabila aplikasi sudah selesai dibuat dan sesuai dengan rancangan sistem yang sudah tertulis. Di tahap ini juga dilakukan analisa permasalahan yang terjadi di aplikasi sebelum aplikasi di luncurkan dan digunakan oleh pengguna.
	\item	Tahap pembuatan laporan \\
	Tahap terakhir ini bertujuan untuk membuat dokumentasi hasil penelitian dalam bentuk laporan Proyek Akhir. Laporan Proyek Akhir akan menjelaskan apapun yang berhubungan dengan perancangan dan pengujian aplikasi.  
	
\end{enumerate}


%-----------------------------------------------------------------------------%
\section{Pembagian Tugas Anggota}
%-----------------------------------------------------------------------------%
Pembagian tugas untuk Proyek Akhir ini adalah sebagai berikut :

\begin{enumerate}
	
\item	Zakaria Wahyu Nur Utomo \\
Peran	: Active View Developer \\
Tanggung Jawab:
\begin{enumerate}
\item	Merancang modul active view
\item	Menyelesaikan modul active view
\end{enumerate}
\item	Elsa Jelista Sari  \\
Peran	: Pasive View Developer \\
Tanggung Jawab: 
\begin{enumerate}
\item	Merancang modul pasive view
\item	Menyelesaikan modul pasive view
\end{enumerate}

\end{enumerate}