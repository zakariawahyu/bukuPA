%-----------------------------------------------------------------------------%
\chapter{\babSatu}
%-----------------------------------------------------------------------------%
%-----------------------------------------------------------------------------%
\section{Latar Belakang}
%-----------------------------------------------------------------------------%
Keberlangsungan perguruan tinggi Universitas Telkom tak akan lepas dari peran orang-orang yang bekerja didalamnya. Dengan struktur organisasi yang kompleks, menjadikan Universitas Telkom terdepan dibidangnya. Setiap pemegang jabatan memiliki peran yang penting dalam menunjang visi dan misi Universitas Telkom untuk mencapai tujuannya. Oleh karena itu, setiap orang yang terpilih untuk memegang jabatan penting di perguruan tinggi ini pasti memiliki kompetensi yang sesuai dengan apa yang diharapkan. Orang-orang tersebut dapat terpilih melalui tahap seleksi yang panjang, agar perguruan tinggi Universitas Telkom mendapatkan orang-orang terbaik untuk menjalankan tugasnya.

Banyak model seleksi yang dilakukan untuk menilai seseorang terutama ketika perusahaan mencari seorang pemimpin dan staf, diantaranya dengan melakukan \textit{assessment center} dan mengisi formulir penilaian untuk setiap kandidat yang akan dicalonkan sebagai pemimpin dan staf. Banyak prosedur serta ketentuan yang harus dimiliki oleh calon pemimpin dan staf, baik itu manajer, kepala bagian, kepala urusan, sekretaris atau staf. Setiap orang yang terpilih berarti telah memenuhi ketentuan yang sudah ditetapkan perusahaan. Ketentuan dibuat berdasarkan kompetensi setiap bagian yang disusun dalam kamus kompetensi perusahaan. Melalui kamus kompetensi tersebut juga dapat dijadikan sebagai pedoman untuk bagian sumber daya manusia dalam mencari pegawai yang berpotensi tinggi demi keberlangsungan perusahaan \cite{nicho}.

Masalah yang paling banyak dijumpai pada suatu perusahaan yaitu berkaitan dengan pencarian kandidat yang sesuai dengan \textit{job description}. Banyak kandidat memiliki \textit{skill} yang sama dengan kandidat lainnya, namun perusahaan mencari kandidat sesuai dengan \textit{requirement} yang telah ditetapkan oleh perusahaan. Dengan begitu, proses \textit{filtering} akan membutuhkan waktu yang lama, jika kandidat tidak cepat ditemukan sesuai \textit{requirement} yang ada. Untuk mengatasi permasalahan tersebut, proses \textit{filtering} akan dipindahkan dengan aplikasi “\textbf{SiPJabS : Sistem Pengawakan Jabatan Struktural}”, yang diharapkan dapat membantu penemuan kandidat yang sesuai dengan \textit{requirement} yang sudah ditetapkan oleh perusahaan.
\\

Terdapat permasalahan belum adanya proses mekanisme penentuan kandidat yang tepat, apabila terdapat posisi yang kosong. Maka tidak adanya data pegawai yang akan dijadikan kandidat untuk mengisi posisi yang kosong tersebut, sehingga perlu dirancang sistem yang mampu mengidentifikasi sesuai kebutuhan posisi yang diinginkan. Yang kemudian hasilnya akan disesuaikan dengan \textit{job description} yang dibutuhkan oleh perusahaan. Dengan cara ini, manajer di perusahaan dapat menentukan profil pegawai yang tepat sesuai \textit{requirement} yang sudah ditetapkan oleh perusahaan.

Untuk mengatasi permasalahan di atas, kami merancang Proyek Akhir ini dengan membuat sistem pengawakan jabatan struktural yang bertujuan untuk untuk mengelola proses pencarian kandidat yang akan dicalonkan sebagai pemimpin dan staf, agar perusahaan memperoleh kandidat yang memenuhi ketentuan yang sudah ditetapkan perusahaan dengan menerapkan penilaian yang lebih lengkap dan adil. Diharapkan dengan adanya sistem ini, manajer mampu menganalisa kebutuhan program pengembangan kompetensi sumber daya manusia yang lebih baik di masa depan.


%-----------------------------------------------------------------------------%
\section{Perumusan Masalah}
%-----------------------------------------------------------------------------%
Berdasarkan latar belakang di atas, maka rumusan masalah yang akan dibahas adalah sebagai berikut:
\begin{enumerate}
\item Bagaimana proses pengisian posisi jabatan yang tepat?
\item Bagaimana cara pencarian kandidat yang sesuai dengan \textit{requirement}?
\item Bagaimana proses \textit{filtering} berjalan efektif?
\end{enumerate}

%-----------------------------------------------------------------------------%
\section{Batasan Permasalahan}
%-----------------------------------------------------------------------------%
Batasan masalah yang terdapat dapat dari perumusan masalah di atas adalah sebagai berikut:

\begin{enumerate}
	\item Aplikasi ini ditunjukan untuk pegawai Direktorat Sumber Daya Manusia Universitas Telkom.
	\item Aplikasi ini dibangun dengan menggunakan sistem berbasis web.
	\item Setiap pemilihan jabatan memiliki parameter berbeda, yang mengacu pada metode penilaian.
\end{enumerate}

%-----------------------------------------------------------------------------%
\section{Tujuan}
%-----------------------------------------------------------------------------%
Tujuan yang ingin dicapai dari perancangan Proyek Akhir ini adalah:

\begin{enumerate}
	\item Membangun aplikasi pengawakan jabatan struktural yang bertujuan untuk mengelola proses pencarian kandidat yang akan dicalonkan sebagai pemimpin dan staf, agar perusahaan memperoleh kandidat yang memenuhi ketentuan yang sudah ditetapkan perusahaan dengan menerapkan penilaian yang lebih lengkap dan adil. Diharapkan dengan adanya sistem ini, manajer mampu menganalisa kebutuhan program pengembangan kompetensi sumber daya manusia yang lebih baik di masa depan.
	
	\item Adanya sistem pengawakan jabatan struktural, pengguna dapat menggunakan aplikasi setiap saat untuk mencari kandidat dan mengisi posisi yang kosong karena menggantikan pekerjaan lama yang telah berhenti dikarenakan pensiun, meninggal, mengundurkan diri atau diberhentikan karena suatu kebijakan tertentu.
\end{enumerate}

%-----------------------------------------------------------------------------%
\section{Metode Penyelesaian Masalah}
%-----------------------------------------------------------------------------%
Metodologi untuk menyelesaikan masalah diatas adalah sebagai berikut:

\begin{enumerate}
	\item Tahap studi literatur \\
	Tahap pertama ini dilakukan dengan cara mencari, menganalisa dan mempelajari informasi yang berhubungan dengan Proyek Akhir. Topik yang berhubungan antara lain: 
	\begin{enumerate}
	\item Data pegawai secara lengkap.
	\item Data jabatan struktural.
	\item Data \textit{requirement} pencarian kandidat. 
	\end{enumerate}
	Serta teori lain yang berhubungan dengan pengembangan aplikasi. Referensi dapat dicari melalui buku, jurnal, \textit{paper}, dan media lainnya baik \textit{daring} maupun \textit{luring}.
	\item	Tahap pencarian dan pengumpulan data \\
	Pencarian dan pengumpulan data yang diperlukan dalam pengembangan aplikasi ini seperti data yang terdapat pada I-GRACIAS dan data yang terdapat pada Direktorat Sumber Daya Manusia.
	\\
	\item	Tahap perancangan sistem \\
	Perancangan sistem aplikasi ini dimulai dengan perancangan mockup atau desain UI/UX aplikasi serta merancang \textit{database} dan kerangka program yang akan digunakan.
	\item Tahap implementasi \\
	Tahap ini dilakukan realisasi dari perancangan sistem yang telah dibuat, seperti membuat \textit{prototype} dan UI dari aplikasi, pembuatan \textit{database} dan aplikasi yang sudah direncanakan pada tahap perancangan sistem.
	\item Tahap pengujian dan analisis \\
	Pengujian dan analisis ini dilakukan apabila aplikasi sudah selesai dibuat serta di \textit{hosting} dan sesuai dengan rancangan sistem yang sudah tertulis. Di tahap ini juga dilakukan analisa permasalahan yang terjadi di aplikasi sebelum aplikasi di luncurkan dan digunakan oleh pengguna.
	\item	Tahap pembuatan laporan \\
	Tahap terakhir ini bertujuan untuk membuat dokumentasi hasil penelitian dalam bentuk laporan Proyek Akhir. Laporan Proyek Akhir akan menjelaskan apapun yang berhubungan dengan perancangan dan pengujian aplikasi.  
	
\end{enumerate}


%-----------------------------------------------------------------------------%
\section{Pembagian Tugas Anggota}
%-----------------------------------------------------------------------------%
Pembagian tugas untuk Proyek Akhir ini adalah sebagai berikut:

\begin{enumerate}
	
\item	Zakaria Wahyu Nur Utomo \\
Peran	: \textit{Back End Developer} dan \textit{Database} \\
Tanggung Jawab:
\begin{enumerate}
\item	Merancang dan membuat sistem aplikasi
\item	Pembuatan buku, poster dan vidio promosi
\end{enumerate}
\item	Elsa Jelista Sari  \\
Peran	: \textit{Front End Developer} dan \textit{Analyst} \\
Tanggung Jawab: 
\begin{enumerate}
\item	Pembuatan \textit{user interface / mockup} dan pengujian aplikasi
\item	Pembuatan buku, jurnal, \textit{user manual} dan vidio demo
\end{enumerate}

\end{enumerate}
